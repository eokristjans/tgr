
% Default to the notebook output style

    


% Inherit from the specified cell style.




    
\documentclass[11pt]{article}

    
    
    \usepackage[T1]{fontenc}
    % Nicer default font (+ math font) than Computer Modern for most use cases
    \usepackage{mathpazo}

    % Basic figure setup, for now with no caption control since it's done
    % automatically by Pandoc (which extracts ![](path) syntax from Markdown).
    \usepackage{graphicx}
    % We will generate all images so they have a width \maxwidth. This means
    % that they will get their normal width if they fit onto the page, but
    % are scaled down if they would overflow the margins.
    \makeatletter
    \def\maxwidth{\ifdim\Gin@nat@width>\linewidth\linewidth
    \else\Gin@nat@width\fi}
    \makeatother
    \let\Oldincludegraphics\includegraphics
    % Set max figure width to be 80% of text width, for now hardcoded.
    \renewcommand{\includegraphics}[1]{\Oldincludegraphics[width=.8\maxwidth]{#1}}
    % Ensure that by default, figures have no caption (until we provide a
    % proper Figure object with a Caption API and a way to capture that
    % in the conversion process - todo).
    \usepackage{caption}
    \DeclareCaptionLabelFormat{nolabel}{}
    \captionsetup{labelformat=nolabel}

    \usepackage{adjustbox} % Used to constrain images to a maximum size 
    \usepackage{xcolor} % Allow colors to be defined
    \usepackage{enumerate} % Needed for markdown enumerations to work
    \usepackage{geometry} % Used to adjust the document margins
    \usepackage{amsmath} % Equations
    \usepackage{amssymb} % Equations
    \usepackage{textcomp} % defines textquotesingle
    % Hack from http://tex.stackexchange.com/a/47451/13684:
    \AtBeginDocument{%
        \def\PYZsq{\textquotesingle}% Upright quotes in Pygmentized code
    }
    \usepackage{upquote} % Upright quotes for verbatim code
    \usepackage{eurosym} % defines \euro
    \usepackage[mathletters]{ucs} % Extended unicode (utf-8) support
    \usepackage[utf8x]{inputenc} % Allow utf-8 characters in the tex document
    \usepackage{fancyvrb} % verbatim replacement that allows latex
    \usepackage{grffile} % extends the file name processing of package graphics 
                         % to support a larger range 
    % The hyperref package gives us a pdf with properly built
    % internal navigation ('pdf bookmarks' for the table of contents,
    % internal cross-reference links, web links for URLs, etc.)
    \usepackage{hyperref}
    \usepackage{longtable} % longtable support required by pandoc >1.10
    \usepackage{booktabs}  % table support for pandoc > 1.12.2
    \usepackage[inline]{enumitem} % IRkernel/repr support (it uses the enumerate* environment)
    \usepackage[normalem]{ulem} % ulem is needed to support strikethroughs (\sout)
                                % normalem makes italics be italics, not underlines
    

    
    
    % Colors for the hyperref package
    \definecolor{urlcolor}{rgb}{0,.145,.698}
    \definecolor{linkcolor}{rgb}{.71,0.21,0.01}
    \definecolor{citecolor}{rgb}{.12,.54,.11}

    % ANSI colors
    \definecolor{ansi-black}{HTML}{3E424D}
    \definecolor{ansi-black-intense}{HTML}{282C36}
    \definecolor{ansi-red}{HTML}{E75C58}
    \definecolor{ansi-red-intense}{HTML}{B22B31}
    \definecolor{ansi-green}{HTML}{00A250}
    \definecolor{ansi-green-intense}{HTML}{007427}
    \definecolor{ansi-yellow}{HTML}{DDB62B}
    \definecolor{ansi-yellow-intense}{HTML}{B27D12}
    \definecolor{ansi-blue}{HTML}{208FFB}
    \definecolor{ansi-blue-intense}{HTML}{0065CA}
    \definecolor{ansi-magenta}{HTML}{D160C4}
    \definecolor{ansi-magenta-intense}{HTML}{A03196}
    \definecolor{ansi-cyan}{HTML}{60C6C8}
    \definecolor{ansi-cyan-intense}{HTML}{258F8F}
    \definecolor{ansi-white}{HTML}{C5C1B4}
    \definecolor{ansi-white-intense}{HTML}{A1A6B2}

    % commands and environments needed by pandoc snippets
    % extracted from the output of `pandoc -s`
    \providecommand{\tightlist}{%
      \setlength{\itemsep}{0pt}\setlength{\parskip}{0pt}}
    \DefineVerbatimEnvironment{Highlighting}{Verbatim}{commandchars=\\\{\}}
    % Add ',fontsize=\small' for more characters per line
    \newenvironment{Shaded}{}{}
    \newcommand{\KeywordTok}[1]{\textcolor[rgb]{0.00,0.44,0.13}{\textbf{{#1}}}}
    \newcommand{\DataTypeTok}[1]{\textcolor[rgb]{0.56,0.13,0.00}{{#1}}}
    \newcommand{\DecValTok}[1]{\textcolor[rgb]{0.25,0.63,0.44}{{#1}}}
    \newcommand{\BaseNTok}[1]{\textcolor[rgb]{0.25,0.63,0.44}{{#1}}}
    \newcommand{\FloatTok}[1]{\textcolor[rgb]{0.25,0.63,0.44}{{#1}}}
    \newcommand{\CharTok}[1]{\textcolor[rgb]{0.25,0.44,0.63}{{#1}}}
    \newcommand{\StringTok}[1]{\textcolor[rgb]{0.25,0.44,0.63}{{#1}}}
    \newcommand{\CommentTok}[1]{\textcolor[rgb]{0.38,0.63,0.69}{\textit{{#1}}}}
    \newcommand{\OtherTok}[1]{\textcolor[rgb]{0.00,0.44,0.13}{{#1}}}
    \newcommand{\AlertTok}[1]{\textcolor[rgb]{1.00,0.00,0.00}{\textbf{{#1}}}}
    \newcommand{\FunctionTok}[1]{\textcolor[rgb]{0.02,0.16,0.49}{{#1}}}
    \newcommand{\RegionMarkerTok}[1]{{#1}}
    \newcommand{\ErrorTok}[1]{\textcolor[rgb]{1.00,0.00,0.00}{\textbf{{#1}}}}
    \newcommand{\NormalTok}[1]{{#1}}
    
    % Additional commands for more recent versions of Pandoc
    \newcommand{\ConstantTok}[1]{\textcolor[rgb]{0.53,0.00,0.00}{{#1}}}
    \newcommand{\SpecialCharTok}[1]{\textcolor[rgb]{0.25,0.44,0.63}{{#1}}}
    \newcommand{\VerbatimStringTok}[1]{\textcolor[rgb]{0.25,0.44,0.63}{{#1}}}
    \newcommand{\SpecialStringTok}[1]{\textcolor[rgb]{0.73,0.40,0.53}{{#1}}}
    \newcommand{\ImportTok}[1]{{#1}}
    \newcommand{\DocumentationTok}[1]{\textcolor[rgb]{0.73,0.13,0.13}{\textit{{#1}}}}
    \newcommand{\AnnotationTok}[1]{\textcolor[rgb]{0.38,0.63,0.69}{\textbf{\textit{{#1}}}}}
    \newcommand{\CommentVarTok}[1]{\textcolor[rgb]{0.38,0.63,0.69}{\textbf{\textit{{#1}}}}}
    \newcommand{\VariableTok}[1]{\textcolor[rgb]{0.10,0.09,0.49}{{#1}}}
    \newcommand{\ControlFlowTok}[1]{\textcolor[rgb]{0.00,0.44,0.13}{\textbf{{#1}}}}
    \newcommand{\OperatorTok}[1]{\textcolor[rgb]{0.40,0.40,0.40}{{#1}}}
    \newcommand{\BuiltInTok}[1]{{#1}}
    \newcommand{\ExtensionTok}[1]{{#1}}
    \newcommand{\PreprocessorTok}[1]{\textcolor[rgb]{0.74,0.48,0.00}{{#1}}}
    \newcommand{\AttributeTok}[1]{\textcolor[rgb]{0.49,0.56,0.16}{{#1}}}
    \newcommand{\InformationTok}[1]{\textcolor[rgb]{0.38,0.63,0.69}{\textbf{\textit{{#1}}}}}
    \newcommand{\WarningTok}[1]{\textcolor[rgb]{0.38,0.63,0.69}{\textbf{\textit{{#1}}}}}
    
    
    % Define a nice break command that doesn't care if a line doesn't already
    % exist.
    \def\br{\hspace*{\fill} \\* }
    % Math Jax compatability definitions
    \def\gt{>}
    \def\lt{<}
    % Document parameters
    \title{v1-eok4-dfb2}
    
    
    

    % Pygments definitions
    
\makeatletter
\def\PY@reset{\let\PY@it=\relax \let\PY@bf=\relax%
    \let\PY@ul=\relax \let\PY@tc=\relax%
    \let\PY@bc=\relax \let\PY@ff=\relax}
\def\PY@tok#1{\csname PY@tok@#1\endcsname}
\def\PY@toks#1+{\ifx\relax#1\empty\else%
    \PY@tok{#1}\expandafter\PY@toks\fi}
\def\PY@do#1{\PY@bc{\PY@tc{\PY@ul{%
    \PY@it{\PY@bf{\PY@ff{#1}}}}}}}
\def\PY#1#2{\PY@reset\PY@toks#1+\relax+\PY@do{#2}}

\expandafter\def\csname PY@tok@w\endcsname{\def\PY@tc##1{\textcolor[rgb]{0.73,0.73,0.73}{##1}}}
\expandafter\def\csname PY@tok@c\endcsname{\let\PY@it=\textit\def\PY@tc##1{\textcolor[rgb]{0.25,0.50,0.50}{##1}}}
\expandafter\def\csname PY@tok@cp\endcsname{\def\PY@tc##1{\textcolor[rgb]{0.74,0.48,0.00}{##1}}}
\expandafter\def\csname PY@tok@k\endcsname{\let\PY@bf=\textbf\def\PY@tc##1{\textcolor[rgb]{0.00,0.50,0.00}{##1}}}
\expandafter\def\csname PY@tok@kp\endcsname{\def\PY@tc##1{\textcolor[rgb]{0.00,0.50,0.00}{##1}}}
\expandafter\def\csname PY@tok@kt\endcsname{\def\PY@tc##1{\textcolor[rgb]{0.69,0.00,0.25}{##1}}}
\expandafter\def\csname PY@tok@o\endcsname{\def\PY@tc##1{\textcolor[rgb]{0.40,0.40,0.40}{##1}}}
\expandafter\def\csname PY@tok@ow\endcsname{\let\PY@bf=\textbf\def\PY@tc##1{\textcolor[rgb]{0.67,0.13,1.00}{##1}}}
\expandafter\def\csname PY@tok@nb\endcsname{\def\PY@tc##1{\textcolor[rgb]{0.00,0.50,0.00}{##1}}}
\expandafter\def\csname PY@tok@nf\endcsname{\def\PY@tc##1{\textcolor[rgb]{0.00,0.00,1.00}{##1}}}
\expandafter\def\csname PY@tok@nc\endcsname{\let\PY@bf=\textbf\def\PY@tc##1{\textcolor[rgb]{0.00,0.00,1.00}{##1}}}
\expandafter\def\csname PY@tok@nn\endcsname{\let\PY@bf=\textbf\def\PY@tc##1{\textcolor[rgb]{0.00,0.00,1.00}{##1}}}
\expandafter\def\csname PY@tok@ne\endcsname{\let\PY@bf=\textbf\def\PY@tc##1{\textcolor[rgb]{0.82,0.25,0.23}{##1}}}
\expandafter\def\csname PY@tok@nv\endcsname{\def\PY@tc##1{\textcolor[rgb]{0.10,0.09,0.49}{##1}}}
\expandafter\def\csname PY@tok@no\endcsname{\def\PY@tc##1{\textcolor[rgb]{0.53,0.00,0.00}{##1}}}
\expandafter\def\csname PY@tok@nl\endcsname{\def\PY@tc##1{\textcolor[rgb]{0.63,0.63,0.00}{##1}}}
\expandafter\def\csname PY@tok@ni\endcsname{\let\PY@bf=\textbf\def\PY@tc##1{\textcolor[rgb]{0.60,0.60,0.60}{##1}}}
\expandafter\def\csname PY@tok@na\endcsname{\def\PY@tc##1{\textcolor[rgb]{0.49,0.56,0.16}{##1}}}
\expandafter\def\csname PY@tok@nt\endcsname{\let\PY@bf=\textbf\def\PY@tc##1{\textcolor[rgb]{0.00,0.50,0.00}{##1}}}
\expandafter\def\csname PY@tok@nd\endcsname{\def\PY@tc##1{\textcolor[rgb]{0.67,0.13,1.00}{##1}}}
\expandafter\def\csname PY@tok@s\endcsname{\def\PY@tc##1{\textcolor[rgb]{0.73,0.13,0.13}{##1}}}
\expandafter\def\csname PY@tok@sd\endcsname{\let\PY@it=\textit\def\PY@tc##1{\textcolor[rgb]{0.73,0.13,0.13}{##1}}}
\expandafter\def\csname PY@tok@si\endcsname{\let\PY@bf=\textbf\def\PY@tc##1{\textcolor[rgb]{0.73,0.40,0.53}{##1}}}
\expandafter\def\csname PY@tok@se\endcsname{\let\PY@bf=\textbf\def\PY@tc##1{\textcolor[rgb]{0.73,0.40,0.13}{##1}}}
\expandafter\def\csname PY@tok@sr\endcsname{\def\PY@tc##1{\textcolor[rgb]{0.73,0.40,0.53}{##1}}}
\expandafter\def\csname PY@tok@ss\endcsname{\def\PY@tc##1{\textcolor[rgb]{0.10,0.09,0.49}{##1}}}
\expandafter\def\csname PY@tok@sx\endcsname{\def\PY@tc##1{\textcolor[rgb]{0.00,0.50,0.00}{##1}}}
\expandafter\def\csname PY@tok@m\endcsname{\def\PY@tc##1{\textcolor[rgb]{0.40,0.40,0.40}{##1}}}
\expandafter\def\csname PY@tok@gh\endcsname{\let\PY@bf=\textbf\def\PY@tc##1{\textcolor[rgb]{0.00,0.00,0.50}{##1}}}
\expandafter\def\csname PY@tok@gu\endcsname{\let\PY@bf=\textbf\def\PY@tc##1{\textcolor[rgb]{0.50,0.00,0.50}{##1}}}
\expandafter\def\csname PY@tok@gd\endcsname{\def\PY@tc##1{\textcolor[rgb]{0.63,0.00,0.00}{##1}}}
\expandafter\def\csname PY@tok@gi\endcsname{\def\PY@tc##1{\textcolor[rgb]{0.00,0.63,0.00}{##1}}}
\expandafter\def\csname PY@tok@gr\endcsname{\def\PY@tc##1{\textcolor[rgb]{1.00,0.00,0.00}{##1}}}
\expandafter\def\csname PY@tok@ge\endcsname{\let\PY@it=\textit}
\expandafter\def\csname PY@tok@gs\endcsname{\let\PY@bf=\textbf}
\expandafter\def\csname PY@tok@gp\endcsname{\let\PY@bf=\textbf\def\PY@tc##1{\textcolor[rgb]{0.00,0.00,0.50}{##1}}}
\expandafter\def\csname PY@tok@go\endcsname{\def\PY@tc##1{\textcolor[rgb]{0.53,0.53,0.53}{##1}}}
\expandafter\def\csname PY@tok@gt\endcsname{\def\PY@tc##1{\textcolor[rgb]{0.00,0.27,0.87}{##1}}}
\expandafter\def\csname PY@tok@err\endcsname{\def\PY@bc##1{\setlength{\fboxsep}{0pt}\fcolorbox[rgb]{1.00,0.00,0.00}{1,1,1}{\strut ##1}}}
\expandafter\def\csname PY@tok@kc\endcsname{\let\PY@bf=\textbf\def\PY@tc##1{\textcolor[rgb]{0.00,0.50,0.00}{##1}}}
\expandafter\def\csname PY@tok@kd\endcsname{\let\PY@bf=\textbf\def\PY@tc##1{\textcolor[rgb]{0.00,0.50,0.00}{##1}}}
\expandafter\def\csname PY@tok@kn\endcsname{\let\PY@bf=\textbf\def\PY@tc##1{\textcolor[rgb]{0.00,0.50,0.00}{##1}}}
\expandafter\def\csname PY@tok@kr\endcsname{\let\PY@bf=\textbf\def\PY@tc##1{\textcolor[rgb]{0.00,0.50,0.00}{##1}}}
\expandafter\def\csname PY@tok@bp\endcsname{\def\PY@tc##1{\textcolor[rgb]{0.00,0.50,0.00}{##1}}}
\expandafter\def\csname PY@tok@fm\endcsname{\def\PY@tc##1{\textcolor[rgb]{0.00,0.00,1.00}{##1}}}
\expandafter\def\csname PY@tok@vc\endcsname{\def\PY@tc##1{\textcolor[rgb]{0.10,0.09,0.49}{##1}}}
\expandafter\def\csname PY@tok@vg\endcsname{\def\PY@tc##1{\textcolor[rgb]{0.10,0.09,0.49}{##1}}}
\expandafter\def\csname PY@tok@vi\endcsname{\def\PY@tc##1{\textcolor[rgb]{0.10,0.09,0.49}{##1}}}
\expandafter\def\csname PY@tok@vm\endcsname{\def\PY@tc##1{\textcolor[rgb]{0.10,0.09,0.49}{##1}}}
\expandafter\def\csname PY@tok@sa\endcsname{\def\PY@tc##1{\textcolor[rgb]{0.73,0.13,0.13}{##1}}}
\expandafter\def\csname PY@tok@sb\endcsname{\def\PY@tc##1{\textcolor[rgb]{0.73,0.13,0.13}{##1}}}
\expandafter\def\csname PY@tok@sc\endcsname{\def\PY@tc##1{\textcolor[rgb]{0.73,0.13,0.13}{##1}}}
\expandafter\def\csname PY@tok@dl\endcsname{\def\PY@tc##1{\textcolor[rgb]{0.73,0.13,0.13}{##1}}}
\expandafter\def\csname PY@tok@s2\endcsname{\def\PY@tc##1{\textcolor[rgb]{0.73,0.13,0.13}{##1}}}
\expandafter\def\csname PY@tok@sh\endcsname{\def\PY@tc##1{\textcolor[rgb]{0.73,0.13,0.13}{##1}}}
\expandafter\def\csname PY@tok@s1\endcsname{\def\PY@tc##1{\textcolor[rgb]{0.73,0.13,0.13}{##1}}}
\expandafter\def\csname PY@tok@mb\endcsname{\def\PY@tc##1{\textcolor[rgb]{0.40,0.40,0.40}{##1}}}
\expandafter\def\csname PY@tok@mf\endcsname{\def\PY@tc##1{\textcolor[rgb]{0.40,0.40,0.40}{##1}}}
\expandafter\def\csname PY@tok@mh\endcsname{\def\PY@tc##1{\textcolor[rgb]{0.40,0.40,0.40}{##1}}}
\expandafter\def\csname PY@tok@mi\endcsname{\def\PY@tc##1{\textcolor[rgb]{0.40,0.40,0.40}{##1}}}
\expandafter\def\csname PY@tok@il\endcsname{\def\PY@tc##1{\textcolor[rgb]{0.40,0.40,0.40}{##1}}}
\expandafter\def\csname PY@tok@mo\endcsname{\def\PY@tc##1{\textcolor[rgb]{0.40,0.40,0.40}{##1}}}
\expandafter\def\csname PY@tok@ch\endcsname{\let\PY@it=\textit\def\PY@tc##1{\textcolor[rgb]{0.25,0.50,0.50}{##1}}}
\expandafter\def\csname PY@tok@cm\endcsname{\let\PY@it=\textit\def\PY@tc##1{\textcolor[rgb]{0.25,0.50,0.50}{##1}}}
\expandafter\def\csname PY@tok@cpf\endcsname{\let\PY@it=\textit\def\PY@tc##1{\textcolor[rgb]{0.25,0.50,0.50}{##1}}}
\expandafter\def\csname PY@tok@c1\endcsname{\let\PY@it=\textit\def\PY@tc##1{\textcolor[rgb]{0.25,0.50,0.50}{##1}}}
\expandafter\def\csname PY@tok@cs\endcsname{\let\PY@it=\textit\def\PY@tc##1{\textcolor[rgb]{0.25,0.50,0.50}{##1}}}

\def\PYZbs{\char`\\}
\def\PYZus{\char`\_}
\def\PYZob{\char`\{}
\def\PYZcb{\char`\}}
\def\PYZca{\char`\^}
\def\PYZam{\char`\&}
\def\PYZlt{\char`\<}
\def\PYZgt{\char`\>}
\def\PYZsh{\char`\#}
\def\PYZpc{\char`\%}
\def\PYZdl{\char`\$}
\def\PYZhy{\char`\-}
\def\PYZsq{\char`\'}
\def\PYZdq{\char`\"}
\def\PYZti{\char`\~}
% for compatibility with earlier versions
\def\PYZat{@}
\def\PYZlb{[}
\def\PYZrb{]}
\makeatother


    % Exact colors from NB
    \definecolor{incolor}{rgb}{0.0, 0.0, 0.5}
    \definecolor{outcolor}{rgb}{0.545, 0.0, 0.0}



    
    % Prevent overflowing lines due to hard-to-break entities
    \sloppy 
    % Setup hyperref package
    \hypersetup{
      breaklinks=true,  % so long urls are correctly broken across lines
      colorlinks=true,
      urlcolor=urlcolor,
      linkcolor=linkcolor,
      citecolor=citecolor,
      }
    % Slightly bigger margins than the latex defaults
    
    \geometry{verbose,tmargin=1in,bmargin=1in,lmargin=1in,rmargin=1in}
    
    

    \begin{document}
    
    
    \maketitle
    
    

    
    \section{Verkefni I - Kinematics of the Stewart
Platform}\label{verkefni-i---kinematics-of-the-stewart-platform}

\subsection{Numerical Analysis, 2nd ed., Sauer, Chapter
1}\label{numerical-analysis-2nd-ed.-sauer-chapter-1}

\subsection{Töluleg Greining STAE405}\label{tuxf6luleg-greining-stae405}

\subsubsection{Háskóli Íslands}\label{huxe1skuxf3li-uxedslands}

\subsubsection{Kennari:}\label{kennari}

\begin{itemize}
\tightlist
\item
  Sigurður Freyr Hafstein
\end{itemize}

\subsubsection{Nemendur:}\label{nemendur}

\begin{itemize}
\tightlist
\item
  Erling Óskar Kristjánsson eok4@hi.is
\item
  Davíð Freyr Björnsson dfb2@hi.is
\end{itemize}

    \begin{Verbatim}[commandchars=\\\{\}]
{\color{incolor}In [{\color{incolor} }]:} \PY{c+c1}{\PYZsh{} \PYZhy{}*\PYZhy{} coding: utf\PYZhy{}8 \PYZhy{}*\PYZhy{}}
        \PY{k+kn}{import} \PY{n+nn}{numpy} \PY{k}{as} \PY{n+nn}{np}
        \PY{k+kn}{import} \PY{n+nn}{math}
        \PY{k+kn}{import} \PY{n+nn}{scipy}\PY{n+nn}{.}\PY{n+nn}{optimize} \PY{k}{as} \PY{n+nn}{scOpt}
        \PY{k+kn}{import} \PY{n+nn}{matplotlib}\PY{n+nn}{.}\PY{n+nn}{pyplot} \PY{k}{as} \PY{n+nn}{plt}
        
        \PY{l+s+sd}{\PYZdq{}\PYZdq{}\PYZdq{} }
        \PY{l+s+sd}{Ytri breytur (stikar) sem við skilgreinum sem}
        \PY{l+s+sd}{global eða viðværar breytur }
        \PY{l+s+sd}{\PYZdq{}\PYZdq{}\PYZdq{}}
        
        \PY{c+c1}{\PYZsh{} global x1, x2, y2, L1, L2, L3, gamma, p1, p2, p3 }
        \PY{n}{x1}\PY{o}{=}\PY{n}{y2}   \PY{o}{=} \PY{l+m+mi}{4}     \PY{c+c1}{\PYZsh{} tekid af fig 1.15}
        \PY{n}{x2}      \PY{o}{=} \PY{l+m+mi}{0}     \PY{c+c1}{\PYZsh{} tekid af fig 1.15}
        \PY{n}{L1}      \PY{o}{=} \PY{l+m+mi}{2}     
        \PY{n}{L2} \PY{o}{=} \PY{n}{L3} \PY{o}{=} \PY{n}{np}\PY{o}{.}\PY{n}{sqrt}\PY{p}{(}\PY{l+m+mi}{2}\PY{p}{)}
        \PY{n}{gamma}   \PY{o}{=} \PY{n}{np}\PY{o}{.}\PY{n}{pi}\PY{o}{/}\PY{l+m+mi}{2}
        \PY{n}{p1}\PY{o}{=}\PY{n}{p2}\PY{o}{=}\PY{n}{p3}\PY{o}{=} \PY{n}{np}\PY{o}{.}\PY{n}{sqrt}\PY{p}{(}\PY{l+m+mi}{5}\PY{p}{)}
\end{Verbatim}


    \subsection{Suggested Activity 1}\label{suggested-activity-1}

Parameters \[L_1, L_2, L_3, γ, x_1, x_2, y_2\] are fixed constants,
strut lengths \[p_1, p_2, p_3\] will be known for a given pose.

    \begin{Verbatim}[commandchars=\\\{\}]
{\color{incolor}In [{\color{incolor} }]:} \PY{c+c1}{\PYZsh{} Fallið f skilar f(theta, p1, p2, p3, L1, L2, L3, gamma, x1, x2, y2)}
        \PY{c+c1}{\PYZsh{} fyrir gefin gildi á theta, p1, p2, p3, L1, L2, L3, gamma, x1, x2 og y2}
        \PY{k}{def} \PY{n+nf}{f}\PY{p}{(}\PY{n}{theta}\PY{p}{,} \PY{n}{p1}\PY{p}{,} \PY{n}{p2}\PY{p}{,} \PY{n}{p3}\PY{p}{,} \PY{n}{L1}\PY{p}{,} \PY{n}{L2}\PY{p}{,} \PY{n}{L3}\PY{p}{,} \PY{n}{gamma}\PY{p}{,} \PY{n}{x1}\PY{p}{,} \PY{n}{x2}\PY{p}{,} \PY{n}{y2}\PY{p}{)}\PY{p}{:}
            \PY{n}{A2}      \PY{o}{=} \PY{n}{L3}\PY{o}{*}\PY{n}{np}\PY{o}{.}\PY{n}{cos}\PY{p}{(}\PY{n}{theta}\PY{p}{)}\PY{o}{\PYZhy{}}\PY{n}{x1}
            \PY{n}{B2}      \PY{o}{=} \PY{n}{L3}\PY{o}{*}\PY{n}{np}\PY{o}{.}\PY{n}{sin}\PY{p}{(}\PY{n}{theta}\PY{p}{)}
            \PY{n}{A3}      \PY{o}{=} \PY{n}{L2}\PY{o}{*}\PY{n}{np}\PY{o}{.}\PY{n}{cos}\PY{p}{(}\PY{n}{theta}\PY{o}{+}\PY{n}{gamma}\PY{p}{)}\PY{o}{\PYZhy{}}\PY{n}{x2}
            \PY{n}{B3}      \PY{o}{=} \PY{n}{L2}\PY{o}{*}\PY{n}{np}\PY{o}{.}\PY{n}{sin}\PY{p}{(}\PY{n}{theta}\PY{o}{+}\PY{n}{gamma}\PY{p}{)}\PY{o}{\PYZhy{}}\PY{n}{y2}
            \PY{n}{aNum}    \PY{o}{=} \PY{n}{p2}\PY{o}{*}\PY{o}{*}\PY{l+m+mi}{2}\PY{o}{\PYZhy{}}\PY{n}{p1}\PY{o}{*}\PY{o}{*}\PY{l+m+mi}{2}\PY{o}{\PYZhy{}}\PY{n}{A2}\PY{o}{*}\PY{o}{*}\PY{l+m+mi}{2}\PY{o}{\PYZhy{}}\PY{n}{B2}\PY{o}{*}\PY{o}{*}\PY{l+m+mi}{2}
            \PY{n}{bNum}    \PY{o}{=} \PY{n}{p3}\PY{o}{*}\PY{o}{*}\PY{l+m+mi}{2}\PY{o}{\PYZhy{}}\PY{n}{p1}\PY{o}{*}\PY{o}{*}\PY{l+m+mi}{2}\PY{o}{\PYZhy{}}\PY{n}{A3}\PY{o}{*}\PY{o}{*}\PY{l+m+mi}{2}\PY{o}{\PYZhy{}}\PY{n}{B3}\PY{o}{*}\PY{o}{*}\PY{l+m+mi}{2}
            \PY{n}{D}       \PY{o}{=} \PY{l+m+mi}{2}\PY{o}{*}\PY{p}{(}\PY{n}{A2}\PY{o}{*}\PY{n}{B3}\PY{o}{\PYZhy{}}\PY{n}{B2}\PY{o}{*}\PY{n}{A3}\PY{p}{)}
            \PY{n}{N1}      \PY{o}{=} \PY{n}{B3}\PY{o}{*}\PY{n}{aNum}\PY{o}{\PYZhy{}}\PY{n}{B2}\PY{o}{*}\PY{n}{bNum}
            \PY{n}{N2}      \PY{o}{=}\PY{o}{\PYZhy{}}\PY{n}{A3}\PY{o}{*}\PY{n}{aNum}\PY{o}{+}\PY{n}{A2}\PY{o}{*}\PY{n}{bNum} 
            \PY{k}{return} \PY{n}{N1}\PY{o}{*}\PY{o}{*}\PY{l+m+mi}{2}\PY{o}{+}\PY{n}{N2}\PY{o}{*}\PY{o}{*}\PY{l+m+mi}{2}\PY{o}{\PYZhy{}}\PY{n}{p1}\PY{o}{*}\PY{o}{*}\PY{l+m+mi}{2}\PY{o}{*}\PY{n}{D}\PY{o}{*}\PY{o}{*}\PY{l+m+mi}{2}
\end{Verbatim}


    Látum \(\theta_1 = -\pi/4\) og \(\theta_2 = \pi/4\).

Nú fæst að \(f(\theta_1) =\)

    \begin{Verbatim}[commandchars=\\\{\}]
{\color{incolor}In [{\color{incolor} }]:} \PY{n+nb}{print}\PY{p}{(}\PY{n}{f}\PY{p}{(}\PY{n}{np}\PY{o}{.}\PY{n}{pi}\PY{o}{/}\PY{l+m+mi}{4}\PY{p}{,} \PY{n}{p1}\PY{p}{,} \PY{n}{p2}\PY{p}{,} \PY{n}{p3}\PY{p}{,} \PY{n}{L1}\PY{p}{,} \PY{n}{L2}\PY{p}{,} \PY{n}{L3}\PY{p}{,} \PY{n}{gamma}\PY{p}{,} \PY{n}{x1}\PY{p}{,} \PY{n}{x2}\PY{p}{,} \PY{n}{y2}\PY{p}{)}\PY{p}{)}
\end{Verbatim}


    \begin{Verbatim}[commandchars=\\\{\}]
-4.547473508864641e-13

    \end{Verbatim}

    og \(f(\theta_2) =\)

    \begin{Verbatim}[commandchars=\\\{\}]
{\color{incolor}In [{\color{incolor} }]:} \PY{n+nb}{print}\PY{p}{(}\PY{n}{f}\PY{p}{(}\PY{o}{\PYZhy{}}\PY{n}{np}\PY{o}{.}\PY{n}{pi}\PY{o}{/}\PY{l+m+mi}{4}\PY{p}{,} \PY{n}{p1}\PY{p}{,} \PY{n}{p2}\PY{p}{,} \PY{n}{p3}\PY{p}{,} \PY{n}{L1}\PY{p}{,} \PY{n}{L2}\PY{p}{,} \PY{n}{L3}\PY{p}{,} \PY{n}{gamma}\PY{p}{,} \PY{n}{x1}\PY{p}{,} \PY{n}{x2}\PY{p}{,} \PY{n}{y2}\PY{p}{)}\PY{p}{)}     
\end{Verbatim}


    \begin{Verbatim}[commandchars=\\\{\}]
-4.547473508864641e-13

    \end{Verbatim}

    Eins og sést þá eru þessar tölur mjög nálægt núlli, innan skekkju
\(10^{-12}\), svo allt lítur vel út hingað til.

\subsection{Suggested Activity 2}\label{suggested-activity-2}

Plot f(θ) on \([−π,π]\)

    \begin{Verbatim}[commandchars=\\\{\}]
{\color{incolor}In [{\color{incolor} }]:} \PY{k}{def} \PY{n+nf}{fPlot}\PY{p}{(}\PY{n}{thetaL}\PY{p}{,} \PY{n}{thetaH}\PY{p}{,} \PY{n}{fileName}\PY{p}{,} \PY{n}{graphTitle}\PY{p}{,} 
                  \PY{n}{minGraphVal}\PY{p}{,} \PY{n}{maxGraphVal}\PY{p}{,} \PY{n}{interval}\PY{p}{)}\PY{p}{:}
                \PY{c+c1}{\PYZsh{} Setjum bilið á theta sem við viljum}
                \PY{c+c1}{\PYZsh{} teikna f(theta) fyrir}
                \PY{n}{theRange} \PY{o}{=} \PY{n}{np}\PY{o}{.}\PY{n}{arange}\PY{p}{(}\PY{n}{thetaL}\PY{p}{,} \PY{n}{thetaH}\PY{p}{,} \PY{l+m+mf}{0.1}\PY{p}{)}
                \PY{c+c1}{\PYZsh{} Finnum hámarks og lágmarks gildi f á því bili}
                \PY{n}{maxFunVal} \PY{o}{=} \PY{n+nb}{max}\PY{p}{(}\PY{n}{f}\PY{p}{(}\PY{n}{theRange}\PY{p}{,} \PY{n}{p1}\PY{p}{,} \PY{n}{p2}\PY{p}{,} \PY{n}{p3}\PY{p}{,} 
                                  \PY{n}{L1}\PY{p}{,} \PY{n}{L2}\PY{p}{,} \PY{n}{L3}\PY{p}{,} \PY{n}{gamma}\PY{p}{,} \PY{n}{x1}\PY{p}{,} \PY{n}{x2}\PY{p}{,} \PY{n}{y2}\PY{p}{)}\PY{p}{)}
                \PY{n}{minFunVal} \PY{o}{=} \PY{n+nb}{min}\PY{p}{(}\PY{n}{f}\PY{p}{(}\PY{n}{theRange}\PY{p}{,} \PY{n}{p1}\PY{p}{,} \PY{n}{p2}\PY{p}{,} \PY{n}{p3}\PY{p}{,} 
                                  \PY{n}{L1}\PY{p}{,} \PY{n}{L2}\PY{p}{,} \PY{n}{L3}\PY{p}{,} \PY{n}{gamma}\PY{p}{,} \PY{n}{x1}\PY{p}{,} \PY{n}{x2}\PY{p}{,} \PY{n}{y2}\PY{p}{)}\PY{p}{)}
                \PY{c+c1}{\PYZsh{} Upphafsstillum myndina og ásana}
                \PY{n}{figure}\PY{p}{,} \PY{n}{axis} \PY{o}{=} \PY{n}{plt}\PY{o}{.}\PY{n}{subplots}\PY{p}{(}\PY{n}{figsize}\PY{o}{=}\PY{p}{(}\PY{l+m+mi}{12}\PY{p}{,} \PY{l+m+mi}{6}\PY{p}{)}\PY{p}{)}
                \PY{c+c1}{\PYZsh{} Teiknum grafið}
                \PY{n}{axis}\PY{o}{.}\PY{n}{plot}\PY{p}{(}\PY{n}{theRange}\PY{p}{,} \PY{n}{f}\PY{p}{(}\PY{n}{theRange}\PY{p}{,} \PY{n}{p1}\PY{p}{,} \PY{n}{p2}\PY{p}{,} \PY{n}{p3}\PY{p}{,} 
                                      \PY{n}{L1}\PY{p}{,} \PY{n}{L2}\PY{p}{,} \PY{n}{L3}\PY{p}{,} \PY{n}{gamma}\PY{p}{,} \PY{n}{x1}\PY{p}{,} \PY{n}{x2}\PY{p}{,} \PY{n}{y2}\PY{p}{)}\PY{p}{)}
                \PY{c+c1}{\PYZsh{} Setjum titil á grafið og ásana}
                \PY{n}{axis}\PY{o}{.}\PY{n}{set}\PY{p}{(}\PY{n}{xlabel}\PY{o}{=}\PY{l+s+sa}{r}\PY{l+s+s2}{\PYZdq{}}\PY{l+s+s2}{\PYZdl{}}\PY{l+s+s2}{\PYZob{}}\PY{l+s+s2}{\PYZbs{}}\PY{l+s+s2}{Theta\PYZcb{}\PYZdl{} [radians]}\PY{l+s+s2}{\PYZdq{}}\PY{p}{,} 
                        \PY{n}{ylabel}\PY{o}{=}\PY{l+s+s2}{\PYZdq{}}\PY{l+s+s2}{\PYZdl{}f(}\PY{l+s+s2}{\PYZob{}}\PY{l+s+s2}{\PYZbs{}}\PY{l+s+s2}{Theta\PYZcb{})\PYZdl{}}\PY{l+s+s2}{\PYZdq{}}\PY{p}{,}
                        \PY{n}{title} \PY{o}{=} \PY{n}{graphTitle}\PY{p}{)}
                \PY{c+c1}{\PYZsh{} Setjum hæsta og lægsta gildið á y\PYZhy{}ás}
                \PY{n}{axis}\PY{o}{.}\PY{n}{set\PYZus{}ylim}\PY{p}{(}\PY{p}{[}\PY{n}{minGraphVal}\PY{p}{,}\PY{n}{maxGraphVal}\PY{p}{]}\PY{p}{)}
                \PY{n}{axis}\PY{o}{.}\PY{n}{grid}\PY{p}{(}\PY{p}{)}
                \PY{c+c1}{\PYZsh{} Vistum myndina}
                \PY{n}{fileString} \PY{o}{=} \PY{n}{fileName} \PY{o}{+} \PY{l+s+s2}{\PYZdq{}}\PY{l+s+s2}{.png}\PY{l+s+s2}{\PYZdq{}}
                \PY{n}{figure}\PY{o}{.}\PY{n}{savefig}\PY{p}{(}\PY{n}{fileString}\PY{p}{)}
                
                \PY{c+c1}{\PYZsh{} Setjum minnsta sýnilega bilið á x \PYZhy{} ásnum}
                \PY{c+c1}{\PYZsh{} og y \PYZhy{} ásnum}
                \PY{n}{plt}\PY{o}{.}\PY{n}{xticks}\PY{p}{(}\PY{n}{np}\PY{o}{.}\PY{n}{arange}\PY{p}{(}\PY{n}{thetaL}\PY{p}{,} \PY{n}{thetaH}\PY{p}{,} \PY{l+m+mf}{0.5}\PY{p}{)}\PY{p}{)}
                \PY{n}{plt}\PY{o}{.}\PY{n}{yticks}\PY{p}{(}\PY{n}{np}\PY{o}{.}\PY{n}{arange}\PY{p}{(}\PY{n}{minGraphVal}\PY{p}{,}\PY{n}{maxGraphVal}\PY{p}{,}\PY{n}{interval}\PY{p}{)}\PY{p}{)}
                \PY{n}{plt}\PY{o}{.}\PY{n}{grid}\PY{p}{(}\PY{k+kc}{True}\PY{p}{)}
                \PY{n}{plt}\PY{o}{.}\PY{n}{show}\PY{p}{(}\PY{p}{)}
\end{Verbatim}


    Teiknum \(f(\theta)\) fyrir \(\theta\) frá \(-\pi\) upp í \(\pi\):

    \begin{Verbatim}[commandchars=\\\{\}]
{\color{incolor}In [{\color{incolor} }]:} \PY{n}{fPlot}\PY{p}{(}\PY{o}{\PYZhy{}}\PY{n}{np}\PY{o}{.}\PY{n}{pi}\PY{p}{,} \PY{n}{np}\PY{o}{.}\PY{n}{pi}\PY{p}{,} \PY{l+s+s2}{\PYZdq{}}\PY{l+s+s2}{sa2}\PY{l+s+s2}{\PYZdq{}}\PY{p}{,} \PY{l+s+s1}{\PYZsq{}}\PY{l+s+s1}{Suggested Activity 2}\PY{l+s+s1}{\PYZsq{}}\PY{p}{,} \PY{o}{\PYZhy{}}\PY{l+m+mi}{2500}\PY{p}{,} \PY{l+m+mi}{35000}\PY{p}{,} \PY{l+m+mi}{2500}\PY{p}{)}
\end{Verbatim}


    \begin{center}
    \adjustimage{max size={0.9\linewidth}{0.9\paperheight}}{output_11_0.png}
    \end{center}
    { \hspace*{\fill} \\}
    
    Python fallið fSolver skilar núllstöð \(f\). Þarf upphafságiskun sem
byggir á því að skoða graf af \(f(\theta)\)

    \begin{Verbatim}[commandchars=\\\{\}]
{\color{incolor}In [{\color{incolor} }]:} \PY{k}{def} \PY{n+nf}{fSolver}\PY{p}{(}\PY{n}{f}\PY{p}{,} \PY{n}{thetaGuess}\PY{p}{)}\PY{p}{:} 
                \PY{n}{func} \PY{o}{=} \PY{k}{lambda} \PY{n}{theta} \PY{p}{:} \PY{n}{f}\PY{p}{(}\PY{n}{theta}\PY{p}{,} \PY{n}{p1}\PY{p}{,} \PY{n}{p2}\PY{p}{,} \PY{n}{p3}\PY{p}{,} 
                                        \PY{n}{L1}\PY{p}{,} \PY{n}{L2}\PY{p}{,} \PY{n}{L3}\PY{p}{,} \PY{n}{gamma}\PY{p}{,} \PY{n}{x1}\PY{p}{,} \PY{n}{x2}\PY{p}{,} \PY{n}{y2}\PY{p}{)}
                \PY{n}{thetaSol} \PY{o}{=} \PY{n}{scOpt}\PY{o}{.}\PY{n}{fsolve}\PY{p}{(}\PY{n}{func}\PY{p}{,} \PY{n}{thetaGuess}\PY{p}{)}
                \PY{k}{return} \PY{n}{thetaSol}
\end{Verbatim}


    Skoðum graf af \$ f(\theta)\$ og sjáum þá fyrir hvaða gildi á
\(\theta, f(\theta)\) er nálægt núlli

    \begin{Verbatim}[commandchars=\\\{\}]
{\color{incolor}In [{\color{incolor} }]:} \PY{n}{thetaCalculated} \PY{o}{=} \PY{n}{fSolver}\PY{p}{(}\PY{n}{f}\PY{p}{,} \PY{n}{np}\PY{o}{.}\PY{n}{pi}\PY{o}{/}\PY{l+m+mi}{5}\PY{p}{)}
        \PY{n}{thetaGiven} \PY{o}{=} \PY{n}{np}\PY{o}{.}\PY{n}{pi}\PY{o}{/}\PY{l+m+mi}{4}
        \PY{n+nb}{print}\PY{p}{(}\PY{l+s+s2}{\PYZdq{}}\PY{l+s+s2}{Núllstöð f(theta) sem fékkst með ítrun er:}\PY{l+s+s2}{\PYZdq{}}\PY{p}{,} \PY{n}{thetaCalculated}\PY{p}{)}
        \PY{n+nb}{print}\PY{p}{(}\PY{l+s+s2}{\PYZdq{}}\PY{l+s+s2}{Gefin núllstöð f(theta) er:}\PY{l+s+s2}{\PYZdq{}}\PY{p}{,} \PY{n}{thetaGiven}\PY{p}{)}
        
        \PY{n}{thetaCalculated} \PY{o}{=} \PY{n}{fSolver}\PY{p}{(}\PY{n}{f}\PY{p}{,} \PY{o}{\PYZhy{}}\PY{n}{np}\PY{o}{.}\PY{n}{pi}\PY{o}{/}\PY{l+m+mi}{5}\PY{p}{)}
        \PY{n}{thetaGiven} \PY{o}{=} \PY{o}{\PYZhy{}}\PY{n}{np}\PY{o}{.}\PY{n}{pi}\PY{o}{/}\PY{l+m+mi}{4}
        \PY{n+nb}{print}\PY{p}{(}\PY{l+s+s2}{\PYZdq{}}\PY{l+s+s2}{Núllstöð f(theta) sem fékkst með ítrun er:}\PY{l+s+s2}{\PYZdq{}}\PY{p}{,} \PY{n}{thetaCalculated}\PY{p}{)}
        \PY{n+nb}{print}\PY{p}{(}\PY{l+s+s2}{\PYZdq{}}\PY{l+s+s2}{Gefin núllstöð f(theta) er:}\PY{l+s+s2}{\PYZdq{}}\PY{p}{,} \PY{n}{thetaGiven}\PY{p}{)}
\end{Verbatim}


    \begin{Verbatim}[commandchars=\\\{\}]
Núllstöð f(theta) sem fékkst með ítrun er: [0.78539816]
Gefin núllstöð f(theta) er: 0.7853981633974483
Núllstöð f(theta) sem fékkst með ítrun er: [-0.78539816]
Gefin núllstöð f(theta) er: -0.7853981633974483

    \end{Verbatim}

    \subsection{Suggested Activity 3}\label{suggested-activity-3}

Reproduce Figure 1.15

    Látum

\begin{align*}
xL_2 &:= x + L_2 \cdot \cos(\theta + \gamma)\\
yL_2 &:= y + L_2 \cdot \sin(\theta + \gamma)\\
xL_3 &:= x + L_3 \cdot \cos(\theta + \gamma)\\
yL_3 &:= y + L_3 \cdot \sin(\theta + \gamma)
\end{align*}

    \begin{Verbatim}[commandchars=\\\{\}]
{\color{incolor}In [{\color{incolor} }]:} \PY{c+c1}{\PYZsh{} Teiknar Stewart Platform}
        \PY{k}{def} \PY{n+nf}{plotStewartPlatform}\PY{p}{(}\PY{n}{x}\PY{p}{,} \PY{n}{y}\PY{p}{,} \PY{n}{x1}\PY{p}{,} \PY{n}{x2}\PY{p}{,} \PY{n}{y2}\PY{p}{,} \PY{n}{xL2}\PY{p}{,} \PY{n}{yL2}\PY{p}{,} \PY{n}{xL3}\PY{p}{,} \PY{n}{yL3}\PY{p}{,} 
                                \PY{n}{plotTitle}\PY{o}{=}\PY{l+s+s2}{\PYZdq{}}\PY{l+s+s2}{Vantar titil!}\PY{l+s+s2}{\PYZdq{}}\PY{p}{,} \PY{n}{plotName}\PY{o}{=}\PY{l+s+s2}{\PYZdq{}}\PY{l+s+s2}{Vantar}\PY{l+s+s2}{\PYZdq{}}\PY{p}{)}\PY{p}{:}
                \PY{c+c1}{\PYZsh{} Teiknar þrihyrning. Byrjum lengst til }
                \PY{c+c1}{\PYZsh{} vinstri og forum rettsaelis}
                \PY{c+c1}{\PYZsh{} Her er (x, y) = (1,2), }
                \PY{c+c1}{\PYZsh{}        (xL2, yL2) = (2,3)}
                \PY{c+c1}{\PYZsh{}        (xL3, yL3) = (2,1)}
                \PY{p}{(}\PY{n}{triangleX}\PY{p}{,} \PY{n}{triangleY}\PY{p}{)}  \PY{o}{=} \PY{p}{[}\PY{n}{x}\PY{p}{,} \PY{n}{xL2}\PY{p}{,} \PY{n}{xL3}\PY{p}{,} \PY{n}{x}\PY{p}{]}\PY{p}{,}\PY{p}{[}\PY{n}{y}\PY{p}{,} \PY{n}{yL2}\PY{p}{,} \PY{n}{yL3}\PY{p}{,} \PY{n}{y}\PY{p}{]}
                \PY{n}{fig}\PY{p}{,} \PY{n}{plotObject} \PY{o}{=} \PY{n}{plt}\PY{o}{.}\PY{n}{subplots}\PY{p}{(}\PY{p}{)}
                \PY{n}{plotObject}\PY{o}{.}\PY{n}{plot}\PY{p}{(}\PY{n}{triangleX}\PY{p}{,} \PY{n}{triangleY}\PY{p}{,} \PY{l+s+s1}{\PYZsq{}}\PY{l+s+s1}{b}\PY{l+s+s1}{\PYZsq{}}\PY{p}{,} \PY{n}{linewidth}\PY{o}{=}\PY{l+m+mf}{2.5}\PY{p}{)}
        
                \PY{c+c1}{\PYZsh{} Baetum vid fyrsta akkerinu}
                \PY{c+c1}{\PYZsh{} sem hefur hnit fra (0,0) i (x,y)}
                \PY{n}{firstAnchorX}\PY{p}{,} \PY{n}{firstAnchorY}  \PY{o}{=} \PY{p}{[}\PY{l+m+mi}{0}\PY{p}{,}\PY{n}{x}\PY{p}{]}\PY{p}{,}\PY{p}{[}\PY{l+m+mi}{0}\PY{p}{,}\PY{n}{y}\PY{p}{]}
                \PY{n}{plotObject}\PY{o}{.}\PY{n}{plot}\PY{p}{(}\PY{n}{firstAnchorX}\PY{p}{,} \PY{n}{firstAnchorY}\PY{p}{,} \PY{l+s+s1}{\PYZsq{}}\PY{l+s+s1}{b}\PY{l+s+s1}{\PYZsq{}}\PY{p}{)}
        
                \PY{c+c1}{\PYZsh{} Baetum vid odru akkerinu}
                \PY{c+c1}{\PYZsh{} sem hefur hnit fra (x2, y2)}
                \PY{c+c1}{\PYZsh{} til (xL2, yL2) }
                \PY{n}{secondAnchorX}\PY{p}{,} \PY{n}{secondAnchorY}  \PY{o}{=} \PY{p}{[}\PY{n}{x2}\PY{p}{,}\PY{n}{xL2}\PY{p}{]}\PY{p}{,}\PY{p}{[}\PY{n}{y2}\PY{p}{,}\PY{n}{yL2}\PY{p}{]}
                \PY{n}{plotObject}\PY{o}{.}\PY{n}{plot}\PY{p}{(}\PY{n}{secondAnchorX}\PY{p}{,} \PY{n}{secondAnchorY}\PY{p}{,} \PY{l+s+s1}{\PYZsq{}}\PY{l+s+s1}{b}\PY{l+s+s1}{\PYZsq{}}\PY{p}{)}
        
                \PY{c+c1}{\PYZsh{} Baetum vid thridja akkerinu}
                \PY{c+c1}{\PYZsh{} sem hefur hnit fra (xL3, yL3)}
                \PY{c+c1}{\PYZsh{} til (x1, 0)}
                \PY{n}{thirdAnchorX}\PY{p}{,} \PY{n}{thirdAnchorY}  \PY{o}{=} \PY{p}{[}\PY{n}{x1}\PY{p}{,}\PY{n}{xL3}\PY{p}{]}\PY{p}{,}\PY{p}{[}\PY{l+m+mi}{0}\PY{p}{,}\PY{n}{yL3}\PY{p}{]}
                \PY{n}{plotObject}\PY{o}{.}\PY{n}{plot}\PY{p}{(}\PY{n}{thirdAnchorX}\PY{p}{,} \PY{n}{thirdAnchorY}\PY{p}{,} \PY{l+s+s1}{\PYZsq{}}\PY{l+s+s1}{b}\PY{l+s+s1}{\PYZsq{}}\PY{p}{)}
        
                \PY{c+c1}{\PYZsh{} Baetum vid punktum fyrir }
                \PY{c+c1}{\PYZsh{} akkaeri eitt, tvo og thrju }
                \PY{c+c1}{\PYZsh{} sem hafa hnitin}
                \PY{c+c1}{\PYZsh{} (0,0), (x2, y2) og (x1, 0)}
                \PY{n}{anchorPointsX}\PY{p}{,} \PY{n}{anchorPointsY}  \PY{o}{=} \PY{p}{[}\PY{l+m+mi}{0}\PY{p}{,} \PY{n}{x1}\PY{p}{,} \PY{n}{x2}\PY{p}{]}\PY{p}{,} \PY{p}{[}\PY{l+m+mi}{0}\PY{p}{,} \PY{l+m+mi}{0}\PY{p}{,} \PY{n}{y2}\PY{p}{]}
                \PY{n}{plotObject}\PY{o}{.}\PY{n}{plot}\PY{p}{(}\PY{n}{anchorPointsX}\PY{p}{,} \PY{n}{anchorPointsY}\PY{p}{,} \PY{l+s+s1}{\PYZsq{}}\PY{l+s+s1}{bo}\PY{l+s+s1}{\PYZsq{}}\PY{p}{)}
        
                \PY{c+c1}{\PYZsh{} Baetum vid punktum fyrir}
                \PY{c+c1}{\PYZsh{} hvert horn thrihyrningsins}
                \PY{n}{plotObject}\PY{o}{.}\PY{n}{plot}\PY{p}{(}\PY{n}{triangleX}\PY{p}{,} \PY{n}{triangleY}\PY{p}{,} \PY{l+s+s1}{\PYZsq{}}\PY{l+s+s1}{bo}\PY{l+s+s1}{\PYZsq{}}\PY{p}{)}
        
                \PY{n}{plotObject}\PY{o}{.}\PY{n}{set}\PY{p}{(}\PY{n}{xlabel}\PY{o}{=}\PY{l+s+s2}{\PYZdq{}}\PY{l+s+s2}{x}\PY{l+s+s2}{\PYZdq{}}\PY{p}{,} 
                        \PY{n}{ylabel}\PY{o}{=}\PY{l+s+s2}{\PYZdq{}}\PY{l+s+s2}{y}\PY{l+s+s2}{\PYZdq{}}\PY{p}{,}
                        \PY{n}{title} \PY{o}{=} \PY{n}{plotTitle}\PY{p}{)}
        
                \PY{c+c1}{\PYZsh{} Vistum myndina}
                \PY{n}{plotSaveFile} \PY{o}{=} \PY{n}{plotName} \PY{o}{+} \PY{l+s+s2}{\PYZdq{}}\PY{l+s+s2}{.png}\PY{l+s+s2}{\PYZdq{}}
                \PY{n}{fig}\PY{o}{.}\PY{n}{savefig}\PY{p}{(}\PY{n}{plotSaveFile}\PY{p}{)}
                \PY{n}{plt}\PY{o}{.}\PY{n}{show}\PY{p}{(}\PY{p}{)}
\end{Verbatim}


    Plot SA3(a):

    \begin{Verbatim}[commandchars=\\\{\}]
{\color{incolor}In [{\color{incolor} }]:} \PY{n}{x}\PY{p}{,} \PY{n}{y} \PY{o}{=} \PY{l+m+mi}{1}\PY{p}{,} \PY{l+m+mi}{2}
        \PY{n}{xL2}\PY{p}{,} \PY{n}{yL2} \PY{o}{=} \PY{l+m+mi}{2}\PY{p}{,} \PY{l+m+mi}{3}
        \PY{n}{xL3}\PY{p}{,} \PY{n}{yL3} \PY{o}{=} \PY{l+m+mi}{2}\PY{p}{,} \PY{l+m+mi}{1}
        \PY{n}{plotName} \PY{o}{=} \PY{l+s+s2}{\PYZdq{}}\PY{l+s+s2}{sa3a}\PY{l+s+s2}{\PYZdq{}}
        \PY{n}{plotTitle} \PY{o}{=} \PY{l+s+s1}{\PYZsq{}}\PY{l+s+s1}{Suggested Activity 3 (a)}\PY{l+s+s1}{\PYZsq{}}
        \PY{n}{plotStewartPlatform}\PY{p}{(}\PY{n}{x}\PY{p}{,} \PY{n}{y}\PY{p}{,} \PY{n}{x1}\PY{p}{,} \PY{n}{x2}\PY{p}{,} \PY{n}{y2}\PY{p}{,} 
                            \PY{n}{xL2}\PY{p}{,} \PY{n}{yL2}\PY{p}{,} \PY{n}{xL3}\PY{p}{,} \PY{n}{yL3}\PY{p}{,} \PY{n}{plotTitle}\PY{p}{,} \PY{n}{plotName}\PY{p}{)}
\end{Verbatim}


    \begin{center}
    \adjustimage{max size={0.9\linewidth}{0.9\paperheight}}{output_20_0.png}
    \end{center}
    { \hspace*{\fill} \\}
    
    Plot SA3(b):

    \begin{Verbatim}[commandchars=\\\{\}]
{\color{incolor}In [{\color{incolor} }]:} \PY{l+s+sd}{\PYZsq{}\PYZsq{}\PYZsq{} Plot SA3(b) \PYZsq{}\PYZsq{}\PYZsq{}}
        \PY{n}{x}\PY{p}{,} \PY{n}{y} \PY{o}{=} \PY{l+m+mi}{2}\PY{p}{,} \PY{l+m+mi}{1}
        \PY{n}{xL2}\PY{p}{,} \PY{n}{yL2} \PY{o}{=} \PY{l+m+mi}{1}\PY{p}{,} \PY{l+m+mi}{2}
        \PY{n}{xL3}\PY{p}{,} \PY{n}{yL3} \PY{o}{=} \PY{l+m+mi}{3}\PY{p}{,} \PY{l+m+mi}{2}
        \PY{n}{plotName} \PY{o}{=} \PY{l+s+s2}{\PYZdq{}}\PY{l+s+s2}{sa3b}\PY{l+s+s2}{\PYZdq{}}
        \PY{n}{plotTitle} \PY{o}{=} \PY{l+s+s1}{\PYZsq{}}\PY{l+s+s1}{Suggested Activity 3 (b)}\PY{l+s+s1}{\PYZsq{}}
        \PY{n}{plotStewartPlatform}\PY{p}{(}\PY{n}{x}\PY{p}{,} \PY{n}{y}\PY{p}{,} \PY{n}{x1}\PY{p}{,} \PY{n}{x2}\PY{p}{,} \PY{n}{y2}\PY{p}{,}
                            \PY{n}{xL2}\PY{p}{,} \PY{n}{yL2}\PY{p}{,} \PY{n}{xL3}\PY{p}{,} \PY{n}{yL3}\PY{p}{,} \PY{n}{plotTitle}\PY{p}{,} \PY{n}{plotName}\PY{p}{)}
\end{Verbatim}


    \begin{center}
    \adjustimage{max size={0.9\linewidth}{0.9\paperheight}}{output_22_0.png}
    \end{center}
    { \hspace*{\fill} \\}
    
    \subsection{Suggested Activity 4}\label{suggested-activity-4}

Solve the forward kinematics problem for the planar Stewart platform.
Plot \(f(\theta)\) and then solve \(f(\theta)\) = 0.

Við sjáum að ferilinn \(f(\theta)\) er mjög flatur í kringum
\(f(\theta) = 0\), sem bendir til þess að villumögnunin (e. error
magnification factor) sé mikil á þessu svæði. Þannig þó \(|f(\theta)|\)
sé mjög lág tala fyrir eitthvað \(\theta\) þá getum við samt verið
hlutfallslega mjög langt frá hinni réttu rót \(f(\theta)\). Því þarf
sérstaklega að gæta þess að stop skilyrðið sé ekki of slakt við leit
rótarinnar.

    \begin{Verbatim}[commandchars=\\\{\}]
{\color{incolor}In [{\color{incolor} }]:} \PY{c+c1}{\PYZsh{} Stikar}
        \PY{n}{x1} \PY{o}{=} \PY{l+m+mi}{5} 
        \PY{n}{x2}\PY{p}{,} \PY{n}{y2} \PY{o}{=} \PY{l+m+mi}{0}\PY{p}{,} \PY{l+m+mi}{6}
        \PY{n}{L1} \PY{o}{=} \PY{n}{L3} \PY{o}{=} \PY{l+m+mi}{3} 
        \PY{n}{L2} \PY{o}{=} \PY{l+m+mi}{3}\PY{o}{*}\PY{n}{np}\PY{o}{.}\PY{n}{sqrt}\PY{p}{(}\PY{l+m+mi}{2}\PY{p}{)}
        \PY{n}{gamma} \PY{o}{=} \PY{n}{np}\PY{o}{.}\PY{n}{pi}\PY{o}{/}\PY{l+m+mi}{4}
        \PY{n}{p1} \PY{o}{=} \PY{n}{p2} \PY{o}{=} \PY{l+m+mi}{5}
        \PY{n}{p3} \PY{o}{=} \PY{l+m+mi}{3}
        
        \PY{c+c1}{\PYZsh{} Teiknum f(theta) fyrir theta frá \PYZhy{}pi upp í pi}
        \PY{n}{fPlot}\PY{p}{(}\PY{o}{\PYZhy{}}\PY{n}{np}\PY{o}{.}\PY{n}{pi}\PY{p}{,} \PY{n}{np}\PY{o}{.}\PY{n}{pi}\PY{p}{,} \PY{l+s+s2}{\PYZdq{}}\PY{l+s+s2}{sa4\PYZhy{}0}\PY{l+s+s2}{\PYZdq{}}\PY{p}{,} \PY{l+s+s1}{\PYZsq{}}\PY{l+s+s1}{Suggested Activity 4}\PY{l+s+s1}{\PYZsq{}}\PY{p}{,} \PY{o}{\PYZhy{}}\PY{l+m+mi}{50000}\PY{p}{,} \PY{l+m+mi}{350000}\PY{p}{,} \PY{l+m+mi}{50000}\PY{p}{)}
\end{Verbatim}


    \begin{center}
    \adjustimage{max size={0.9\linewidth}{0.9\paperheight}}{output_24_0.png}
    \end{center}
    { \hspace*{\fill} \\}
    
    Útfrá mynd sést að ágætis upphafsgisk fyrir þessar fjórar núllstöðvar
f(theta) eru -0.7, -0.3, 1.1, 2.1. Finnum núna fjórar núllstöðvar
\(f(\theta)\)

    \begin{Verbatim}[commandchars=\\\{\}]
{\color{incolor}In [{\color{incolor} }]:} \PY{c+c1}{\PYZsh{} Geymum núllstöðvarnar og ágiskanirnar í fylki}
        \PY{n}{thetas} \PY{o}{=} \PY{n}{np}\PY{o}{.}\PY{n}{zeros}\PY{p}{(}\PY{l+m+mi}{4}\PY{p}{,} \PY{n}{dtype}\PY{o}{=}\PY{n+nb}{float}\PY{p}{)}
        \PY{n}{thetasGuess} \PY{o}{=} \PY{p}{[}\PY{o}{\PYZhy{}}\PY{l+m+mf}{0.7}\PY{p}{,} \PY{o}{\PYZhy{}}\PY{l+m+mf}{0.3}\PY{p}{,} \PY{l+m+mf}{1.1}\PY{p}{,} \PY{l+m+mf}{2.1}\PY{p}{]}
        
        \PY{c+c1}{\PYZsh{} Hér fáum við út réttar núllstöðvar}
        \PY{n+nb}{print}\PY{p}{(}\PY{l+s+s2}{\PYZdq{}}\PY{l+s+s2}{Ef theta er raunveruleg núllstöð þá ætti}\PY{l+s+s2}{\PYZdq{}}\PY{p}{)}
        \PY{n+nb}{print}\PY{p}{(}\PY{l+s+s2}{\PYZdq{}}\PY{l+s+s2}{f(theta) að vera mjög nálægt núlli }\PY{l+s+se}{\PYZbs{}n}\PY{l+s+s2}{\PYZdq{}}\PY{p}{)}
        \PY{k}{for} \PY{n}{i} \PY{o+ow}{in} \PY{n+nb}{range}\PY{p}{(}\PY{n+nb}{len}\PY{p}{(}\PY{n}{thetas}\PY{p}{)}\PY{p}{)}\PY{p}{:}
                \PY{n}{thetas}\PY{p}{[}\PY{n}{i}\PY{p}{]} \PY{o}{=} \PY{n}{fSolver}\PY{p}{(}\PY{n}{f}\PY{p}{,} \PY{n}{thetasGuess}\PY{p}{[}\PY{n}{i}\PY{p}{]}\PY{p}{)}
                \PY{n+nb}{print}\PY{p}{(}\PY{l+s+s2}{\PYZdq{}}\PY{l+s+s2}{Nú fæst að theta er:}\PY{l+s+s2}{\PYZdq{}}\PY{p}{,} \PY{n}{thetas}\PY{p}{[}\PY{n}{i}\PY{p}{]}\PY{p}{)}
                \PY{n+nb}{print}\PY{p}{(}\PY{l+s+s2}{\PYZdq{}}\PY{l+s+s2}{Svo er f(theta):}\PY{l+s+s2}{\PYZdq{}}\PY{p}{,} \PY{n}{f}\PY{p}{(}\PY{n}{thetas}\PY{p}{[}\PY{n}{i}\PY{p}{]}\PY{p}{,} \PY{n}{p1}\PY{p}{,} \PY{n}{p2}\PY{p}{,} \PY{n}{p3}\PY{p}{,} \PY{n}{L1}\PY{p}{,} \PY{n}{L2}\PY{p}{,} \PY{n}{L3}\PY{p}{,} \PY{n}{gamma}\PY{p}{,} \PY{n}{x1}\PY{p}{,} \PY{n}{x2}\PY{p}{,} \PY{n}{y2}\PY{p}{)}\PY{p}{,} \PY{l+s+s2}{\PYZdq{}}\PY{l+s+se}{\PYZbs{}n}\PY{l+s+s2}{\PYZdq{}}\PY{p}{)}
        
        \PY{c+c1}{\PYZsh{} Geymir x og y gildin fyrir útreiknaðar núllstöðvar f(theta)}
        \PY{n}{xy} \PY{o}{=} \PY{n}{np}\PY{o}{.}\PY{n}{zeros}\PY{p}{(}\PY{p}{[}\PY{l+m+mi}{4}\PY{p}{,}\PY{l+m+mi}{2}\PY{p}{]}\PY{p}{,} \PY{n}{dtype}\PY{o}{=}\PY{n+nb}{float}\PY{p}{)}
        
        \PY{c+c1}{\PYZsh{} Fallið xyCalc reiknar út gildin á x og y miðað við gefin}
        \PY{c+c1}{\PYZsh{} gildi á theta, p1, p2, p3, L1, L2, L3, gamma, x1, x2 og y2.}
        \PY{k}{def} \PY{n+nf}{xyCalc}\PY{p}{(}\PY{n}{theta}\PY{p}{,} \PY{n}{p1}\PY{p}{,} \PY{n}{p2}\PY{p}{,} \PY{n}{p3}\PY{p}{,} \PY{n}{L1}\PY{p}{,} \PY{n}{L2}\PY{p}{,} \PY{n}{L3}\PY{p}{,} \PY{n}{gamma}\PY{p}{,} \PY{n}{x1}\PY{p}{,} \PY{n}{x2}\PY{p}{,} \PY{n}{y2}\PY{p}{)}\PY{p}{:}
            \PY{n}{A2}      \PY{o}{=} \PY{n}{L3}\PY{o}{*}\PY{n}{np}\PY{o}{.}\PY{n}{cos}\PY{p}{(}\PY{n}{theta}\PY{p}{)}\PY{o}{\PYZhy{}}\PY{n}{x1}
            \PY{n}{B2}      \PY{o}{=} \PY{n}{L3}\PY{o}{*}\PY{n}{np}\PY{o}{.}\PY{n}{sin}\PY{p}{(}\PY{n}{theta}\PY{p}{)}
            \PY{n}{A3}      \PY{o}{=} \PY{n}{L2}\PY{o}{*}\PY{n}{np}\PY{o}{.}\PY{n}{cos}\PY{p}{(}\PY{n}{theta}\PY{o}{+}\PY{n}{gamma}\PY{p}{)}\PY{o}{\PYZhy{}}\PY{n}{x2}
            \PY{n}{B3}      \PY{o}{=} \PY{n}{L2}\PY{o}{*}\PY{n}{np}\PY{o}{.}\PY{n}{sin}\PY{p}{(}\PY{n}{theta}\PY{o}{+}\PY{n}{gamma}\PY{p}{)}\PY{o}{\PYZhy{}}\PY{n}{y2}
            \PY{n}{aNum}    \PY{o}{=} \PY{n}{p2}\PY{o}{*}\PY{o}{*}\PY{l+m+mi}{2}\PY{o}{\PYZhy{}}\PY{n}{p1}\PY{o}{*}\PY{o}{*}\PY{l+m+mi}{2}\PY{o}{\PYZhy{}}\PY{n}{A2}\PY{o}{*}\PY{o}{*}\PY{l+m+mi}{2}\PY{o}{\PYZhy{}}\PY{n}{B2}\PY{o}{*}\PY{o}{*}\PY{l+m+mi}{2}
            \PY{n}{bNum}    \PY{o}{=} \PY{n}{p3}\PY{o}{*}\PY{o}{*}\PY{l+m+mi}{2}\PY{o}{\PYZhy{}}\PY{n}{p1}\PY{o}{*}\PY{o}{*}\PY{l+m+mi}{2}\PY{o}{\PYZhy{}}\PY{n}{A3}\PY{o}{*}\PY{o}{*}\PY{l+m+mi}{2}\PY{o}{\PYZhy{}}\PY{n}{B3}\PY{o}{*}\PY{o}{*}\PY{l+m+mi}{2}
            \PY{n}{D}       \PY{o}{=} \PY{l+m+mi}{2}\PY{o}{*}\PY{p}{(}\PY{n}{A2}\PY{o}{*}\PY{n}{B3}\PY{o}{\PYZhy{}}\PY{n}{B2}\PY{o}{*}\PY{n}{A3}\PY{p}{)}
            \PY{n}{N1}      \PY{o}{=} \PY{n}{B3}\PY{o}{*}\PY{n}{aNum}\PY{o}{\PYZhy{}}\PY{n}{B2}\PY{o}{*}\PY{n}{bNum}
            \PY{n}{N2}      \PY{o}{=}\PY{o}{\PYZhy{}}\PY{n}{A3}\PY{o}{*}\PY{n}{aNum}\PY{o}{+}\PY{n}{A2}\PY{o}{*}\PY{n}{bNum} 
            \PY{n}{x} \PY{o}{=} \PY{n}{N1}\PY{o}{/}\PY{n}{D}
            \PY{n}{y} \PY{o}{=} \PY{n}{N2}\PY{o}{/}\PY{n}{D}
            \PY{k}{return} \PY{p}{[}\PY{n}{x}\PY{p}{,} \PY{n}{y}\PY{p}{]}
        
        \PY{k}{for} \PY{n}{i} \PY{o+ow}{in} \PY{n+nb}{range}\PY{p}{(}\PY{l+m+mi}{0}\PY{p}{,} \PY{l+m+mi}{4}\PY{p}{)}\PY{p}{:} 
                \PY{n}{xy}\PY{p}{[}\PY{n}{i}\PY{p}{]} \PY{o}{=} \PY{n}{xyCalc}\PY{p}{(}\PY{n}{thetas}\PY{p}{[}\PY{n}{i}\PY{p}{]}\PY{p}{,} \PY{n}{p1}\PY{p}{,} \PY{n}{p2}\PY{p}{,} \PY{n}{p3}\PY{p}{,} \PY{n}{L1}\PY{p}{,} \PY{n}{L2}\PY{p}{,} \PY{n}{L3}\PY{p}{,} \PY{n}{gamma}\PY{p}{,} \PY{n}{x1}\PY{p}{,} \PY{n}{x2}\PY{p}{,} \PY{n}{y2}\PY{p}{)}
        
        \PY{c+c1}{\PYZsh{} Útfrá útreiknuðum theta, x og y gildum getum við síðan reiknað xL2, yL2, xL3, yL3}
        \PY{n}{xyL23} \PY{o}{=} \PY{n}{np}\PY{o}{.}\PY{n}{zeros}\PY{p}{(}\PY{p}{[}\PY{l+m+mi}{4}\PY{p}{,}\PY{l+m+mi}{4}\PY{p}{]}\PY{p}{,} \PY{n}{dtype}\PY{o}{=}\PY{n+nb}{float}\PY{p}{)}
        
        \PY{k}{def} \PY{n+nf}{xyLCalc}\PY{p}{(}\PY{n}{theta}\PY{p}{,} \PY{n}{x}\PY{p}{,} \PY{n}{y}\PY{p}{,} \PY{n}{p1}\PY{p}{,} \PY{n}{p2}\PY{p}{,} \PY{n}{p3}\PY{p}{,} \PY{n}{L1}\PY{p}{,} \PY{n}{L2}\PY{p}{,} \PY{n}{L3}\PY{p}{,} \PY{n}{gamma}\PY{p}{,} \PY{n}{x1}\PY{p}{,} \PY{n}{x2}\PY{p}{,} \PY{n}{y2}\PY{p}{)}\PY{p}{:} 
                \PY{n}{xL2} \PY{o}{=} \PY{n}{x} \PY{o}{+} \PY{n}{L2}\PY{o}{*}\PY{n}{np}\PY{o}{.}\PY{n}{cos}\PY{p}{(}\PY{n}{theta} \PY{o}{+} \PY{n}{gamma}\PY{p}{)}
                \PY{n}{yL2} \PY{o}{=} \PY{n}{y} \PY{o}{+} \PY{n}{L2}\PY{o}{*}\PY{n}{np}\PY{o}{.}\PY{n}{sin}\PY{p}{(}\PY{n}{theta} \PY{o}{+} \PY{n}{gamma}\PY{p}{)}
                \PY{n}{xL3} \PY{o}{=} \PY{n}{x} \PY{o}{+} \PY{n}{L3}\PY{o}{*}\PY{n}{np}\PY{o}{.}\PY{n}{cos}\PY{p}{(}\PY{n}{theta}\PY{p}{)}
                \PY{n}{yL3} \PY{o}{=} \PY{n}{y} \PY{o}{+} \PY{n}{L3}\PY{o}{*}\PY{n}{np}\PY{o}{.}\PY{n}{sin}\PY{p}{(}\PY{n}{theta}\PY{p}{)}
                \PY{k}{return} \PY{p}{[}\PY{n}{xL2}\PY{p}{,} \PY{n}{yL2}\PY{p}{,} \PY{n}{xL3}\PY{p}{,} \PY{n}{yL3}\PY{p}{]}
        
        \PY{k}{for} \PY{n}{i} \PY{o+ow}{in} \PY{n+nb}{range}\PY{p}{(}\PY{l+m+mi}{0}\PY{p}{,} \PY{l+m+mi}{4}\PY{p}{)}\PY{p}{:}
                \PY{n}{xyL23}\PY{p}{[}\PY{n}{i}\PY{p}{]} \PY{o}{=} \PY{n}{xyLCalc}\PY{p}{(}\PY{n}{thetas}\PY{p}{[}\PY{n}{i}\PY{p}{]}\PY{p}{,} \PY{n}{xy}\PY{p}{[}\PY{n}{i}\PY{p}{,}\PY{l+m+mi}{0}\PY{p}{]}\PY{p}{,} \PY{n}{xy}\PY{p}{[}\PY{n}{i}\PY{p}{,}\PY{l+m+mi}{1}\PY{p}{]}\PY{p}{,} \PY{n}{p1}\PY{p}{,} \PY{n}{p2}\PY{p}{,} \PY{n}{p3}\PY{p}{,} \PY{n}{L1}\PY{p}{,} \PY{n}{L2}\PY{p}{,} \PY{n}{L3}\PY{p}{,} \PY{n}{gamma}\PY{p}{,} \PY{n}{x1}\PY{p}{,} \PY{n}{x2}\PY{p}{,} \PY{n}{y2}\PY{p}{)}
        
        \PY{c+c1}{\PYZsh{} Teiknum Stewart platform fyrir þessar fjórar núllstöðvar f(theta)}
        \PY{c+c1}{\PYZsh{} og könnum í leiðinni hvort lengdirnar á struts séu p1, p2 og p3}
        \PY{k}{for} \PY{n}{i} \PY{o+ow}{in} \PY{n+nb}{range}\PY{p}{(}\PY{l+m+mi}{0}\PY{p}{,} \PY{l+m+mi}{4}\PY{p}{)}\PY{p}{:}
                \PY{n}{p1Calc} \PY{o}{=} \PY{n}{math}\PY{o}{.}\PY{n}{hypot}\PY{p}{(}\PY{n}{xy}\PY{p}{[}\PY{n}{i}\PY{p}{,}\PY{l+m+mi}{0}\PY{p}{]} \PY{o}{\PYZhy{}} \PY{l+m+mi}{0}\PY{p}{,} \PY{n}{xy}\PY{p}{[}\PY{n}{i}\PY{p}{,}\PY{l+m+mi}{1}\PY{p}{]} \PY{o}{\PYZhy{}} \PY{l+m+mi}{0}\PY{p}{)}
                \PY{n}{p2Calc} \PY{o}{=} \PY{n}{math}\PY{o}{.}\PY{n}{hypot}\PY{p}{(}\PY{n}{xyL23}\PY{p}{[}\PY{n}{i}\PY{p}{,}\PY{l+m+mi}{2}\PY{p}{]} \PY{o}{\PYZhy{}} \PY{n}{x1}\PY{p}{,} \PY{n}{xyL23}\PY{p}{[}\PY{n}{i}\PY{p}{,}\PY{l+m+mi}{3}\PY{p}{]} \PY{o}{\PYZhy{}} \PY{l+m+mi}{0}\PY{p}{)}
                \PY{n}{p3Calc} \PY{o}{=} \PY{n}{math}\PY{o}{.}\PY{n}{hypot}\PY{p}{(}\PY{n}{xyL23}\PY{p}{[}\PY{n}{i}\PY{p}{,}\PY{l+m+mi}{0}\PY{p}{]} \PY{o}{\PYZhy{}} \PY{n}{x2}\PY{p}{,} \PY{n}{xyL23}\PY{p}{[}\PY{n}{i}\PY{p}{,}\PY{l+m+mi}{1}\PY{p}{]} \PY{o}{\PYZhy{}} \PY{n}{y2}\PY{p}{)}
                \PY{n+nb}{print}\PY{p}{(}\PY{l+s+s2}{\PYZdq{}}\PY{l+s+s2}{Fyrir núllstöðina}\PY{l+s+s2}{\PYZdq{}}\PY{p}{,} \PY{n}{thetas}\PY{p}{[}\PY{n}{i}\PY{p}{]}\PY{p}{,} \PY{l+s+s2}{\PYZdq{}}\PY{l+s+s2}{þá fást eftirfarandi niðurstöður}\PY{l+s+s2}{\PYZdq{}}\PY{p}{)}
                \PY{n+nb}{print}\PY{p}{(}\PY{l+s+s2}{\PYZdq{}}\PY{l+s+s2}{Nú er p1 = 5 en útreiknað gildi er:}\PY{l+s+s2}{\PYZdq{}}\PY{p}{,} \PY{n}{p1Calc}\PY{p}{)}
                \PY{n+nb}{print}\PY{p}{(}\PY{l+s+s2}{\PYZdq{}}\PY{l+s+s2}{Nú er p2 = 5 en útreiknað gildi er:}\PY{l+s+s2}{\PYZdq{}}\PY{p}{,} \PY{n}{p2Calc}\PY{p}{)}
                \PY{n+nb}{print}\PY{p}{(}\PY{l+s+s2}{\PYZdq{}}\PY{l+s+s2}{Nú er p3 = 3 en útreiknað gildi er:}\PY{l+s+s2}{\PYZdq{}}\PY{p}{,} \PY{n}{p3Calc}\PY{p}{)}
                \PY{n+nb}{print}\PY{p}{(}\PY{l+s+s2}{\PYZdq{}}\PY{l+s+se}{\PYZbs{}n}\PY{l+s+s2}{\PYZdq{}}\PY{p}{)}
                \PY{n}{pName} \PY{o}{=} \PY{l+s+s2}{\PYZdq{}}\PY{l+s+s2}{sa4\PYZhy{}}\PY{l+s+s2}{\PYZdq{}} \PY{o}{+} \PY{n+nb}{str}\PY{p}{(}\PY{n}{i}\PY{o}{+}\PY{l+m+mi}{1}\PY{p}{)}
                \PY{n}{plotStewartPlatform}\PY{p}{(}\PY{n}{xy}\PY{p}{[}\PY{n}{i}\PY{p}{,}\PY{l+m+mi}{0}\PY{p}{]}\PY{p}{,} \PY{n}{xy}\PY{p}{[}\PY{n}{i}\PY{p}{,}\PY{l+m+mi}{1}\PY{p}{]}\PY{p}{,} \PY{n}{x1}\PY{p}{,} \PY{n}{x2}\PY{p}{,} \PY{n}{y2}\PY{p}{,} \PY{n}{xyL23}\PY{p}{[}\PY{n}{i}\PY{p}{,} \PY{l+m+mi}{0}\PY{p}{]}\PY{p}{,} \PY{n}{xyL23}\PY{p}{[}\PY{n}{i}\PY{p}{,} \PY{l+m+mi}{1}\PY{p}{]}\PY{p}{,} \PY{n}{xyL23}\PY{p}{[}\PY{n}{i}\PY{p}{,} \PY{l+m+mi}{2}\PY{p}{]}\PY{p}{,} \PY{n}{xyL23}\PY{p}{[}\PY{n}{i}\PY{p}{,} \PY{l+m+mi}{3}\PY{p}{]}\PY{p}{,} \PY{l+s+s2}{\PYZdq{}}\PY{l+s+s2}{Suggested Activity 4}\PY{l+s+s2}{\PYZdq{}}\PY{p}{,} \PY{n}{pName}\PY{p}{)}
\end{Verbatim}


    \begin{Verbatim}[commandchars=\\\{\}]
Ef theta er raunveruleg núllstöð þá ætti
f(theta) að vera mjög nálægt núlli 

Nú fæst að theta er: -0.7208492044603826
Svo er f(theta): -1.3096723705530167e-10 

Nú fæst að theta er: -0.33100518428386466
Svo er f(theta): 4.9112713895738125e-11 

Nú fæst að theta er: 1.1436855178213738
Svo er f(theta): 5.4569682106375694e-12 

Nú fæst að theta er: 2.115909014086863
Svo er f(theta): 3.710738383233547e-08 

Fyrir núllstöðina -0.7208492044603826 þá fást eftirfarandi niðurstöður
Nú er p1 = 5 en útreiknað gildi er: 4.999999999999995
Nú er p2 = 5 en útreiknað gildi er: 4.999999999999995
Nú er p3 = 3 en útreiknað gildi er: 2.999999999999991



    \end{Verbatim}

    \begin{center}
    \adjustimage{max size={0.9\linewidth}{0.9\paperheight}}{output_26_1.png}
    \end{center}
    { \hspace*{\fill} \\}
    
    \begin{Verbatim}[commandchars=\\\{\}]
Fyrir núllstöðina -0.33100518428386466 þá fást eftirfarandi niðurstöður
Nú er p1 = 5 en útreiknað gildi er: 5.000000000000008
Nú er p2 = 5 en útreiknað gildi er: 5.000000000000008
Nú er p3 = 3 en útreiknað gildi er: 3.0000000000000124



    \end{Verbatim}

    \begin{center}
    \adjustimage{max size={0.9\linewidth}{0.9\paperheight}}{output_26_3.png}
    \end{center}
    { \hspace*{\fill} \\}
    
    \begin{Verbatim}[commandchars=\\\{\}]
Fyrir núllstöðina 1.1436855178213738 þá fást eftirfarandi niðurstöður
Nú er p1 = 5 en útreiknað gildi er: 5.000000000000001
Nú er p2 = 5 en útreiknað gildi er: 5.0
Nú er p3 = 3 en útreiknað gildi er: 3.0000000000000004



    \end{Verbatim}

    \begin{center}
    \adjustimage{max size={0.9\linewidth}{0.9\paperheight}}{output_26_5.png}
    \end{center}
    { \hspace*{\fill} \\}
    
    \begin{Verbatim}[commandchars=\\\{\}]
Fyrir núllstöðina 2.115909014086863 þá fást eftirfarandi niðurstöður
Nú er p1 = 5 en útreiknað gildi er: 5.000000000000496
Nú er p2 = 5 en útreiknað gildi er: 5.000000000000496
Nú er p3 = 3 en útreiknað gildi er: 3.000000000000825



    \end{Verbatim}

    \begin{center}
    \adjustimage{max size={0.9\linewidth}{0.9\paperheight}}{output_26_7.png}
    \end{center}
    { \hspace*{\fill} \\}
    
    \subsection{Suggested Activity 5}\label{suggested-activity-5}

Change strut length to \(p_2 = 7\) and re-solve the problem. For these
parameters, there are six poses.

    \begin{Verbatim}[commandchars=\\\{\}]
{\color{incolor}In [{\color{incolor} }]:} \PY{c+c1}{\PYZsh{} Change the value of p2}
        \PY{n}{p2} \PY{o}{=} \PY{l+m+mi}{7}
        
        \PY{c+c1}{\PYZsh{} Teiknum f(theta) fyrir theta frá \PYZhy{}pi upp í pi}
        \PY{n}{fPlot}\PY{p}{(}\PY{o}{\PYZhy{}}\PY{n}{np}\PY{o}{.}\PY{n}{pi}\PY{p}{,} \PY{n}{np}\PY{o}{.}\PY{n}{pi}\PY{p}{,} \PY{l+s+s2}{\PYZdq{}}\PY{l+s+s2}{sa5\PYZhy{}0}\PY{l+s+s2}{\PYZdq{}}\PY{p}{,} \PY{l+s+s1}{\PYZsq{}}\PY{l+s+s1}{Suggested Activity 5}\PY{l+s+s1}{\PYZsq{}}\PY{p}{,} \PY{o}{\PYZhy{}}\PY{l+m+mi}{50000}\PY{p}{,} \PY{l+m+mi}{250000}\PY{p}{,} \PY{l+m+mi}{50000}\PY{p}{)}
        
        \PY{c+c1}{\PYZsh{} Finnum núna sex núllstöðvar f(theta)}
        \PY{c+c1}{\PYZsh{} Hér er ferillinn mjög flatur í kringum fimm}
        \PY{c+c1}{\PYZsh{} af sex núllstöðvunum}
        \PY{n}{thetas2} \PY{o}{=} \PY{n}{np}\PY{o}{.}\PY{n}{zeros}\PY{p}{(}\PY{l+m+mi}{6}\PY{p}{,} \PY{n}{dtype}\PY{o}{=}\PY{n+nb}{float}\PY{p}{)}
        
        \PY{c+c1}{\PYZsh{} Útfrá mynd sést að ágætis upphafsgisk fyrir}
        \PY{c+c1}{\PYZsh{} þessar sex núllstöðvar f(theta) eru}
        \PY{c+c1}{\PYZsh{} \PYZhy{}0.7, \PYZhy{}0.4, 0, 0.42, 1, 2.5}
        \PY{n}{thetasGuess2} \PY{o}{=} \PY{p}{[}\PY{o}{\PYZhy{}}\PY{l+m+mf}{0.7}\PY{p}{,} \PY{o}{\PYZhy{}}\PY{l+m+mf}{0.4}\PY{p}{,} \PY{l+m+mi}{0}\PY{p}{,} \PY{l+m+mf}{0.42}\PY{p}{,} \PY{l+m+mi}{1}\PY{p}{,} \PY{l+m+mf}{2.5}\PY{p}{]}
        
        \PY{c+c1}{\PYZsh{} Hér fáum við út réttar núllstöðvar}
        \PY{n+nb}{print}\PY{p}{(}\PY{l+s+s2}{\PYZdq{}}\PY{l+s+s2}{Ef theta er raunveruleg núllstöð þá ætti}\PY{l+s+s2}{\PYZdq{}}\PY{p}{)}
        \PY{n+nb}{print}\PY{p}{(}\PY{l+s+s2}{\PYZdq{}}\PY{l+s+s2}{f(theta) að vera mjög nálægt núlli }\PY{l+s+se}{\PYZbs{}n}\PY{l+s+s2}{\PYZdq{}}\PY{p}{)}
        \PY{k}{for} \PY{n}{i} \PY{o+ow}{in} \PY{n+nb}{range}\PY{p}{(}\PY{l+m+mi}{0}\PY{p}{,}\PY{n+nb}{len}\PY{p}{(}\PY{n}{thetas2}\PY{p}{)}\PY{p}{)}\PY{p}{:}
                \PY{n}{thetas2}\PY{p}{[}\PY{n}{i}\PY{p}{]} \PY{o}{=} \PY{n}{fSolver}\PY{p}{(}\PY{n}{f}\PY{p}{,} \PY{n}{thetasGuess2}\PY{p}{[}\PY{n}{i}\PY{p}{]}\PY{p}{)}
                \PY{n+nb}{print}\PY{p}{(}\PY{l+s+s2}{\PYZdq{}}\PY{l+s+s2}{Nú fæst að theta er:}\PY{l+s+s2}{\PYZdq{}}\PY{p}{,} \PY{n}{thetas2}\PY{p}{[}\PY{n}{i}\PY{p}{]}\PY{p}{)}
                \PY{n+nb}{print}\PY{p}{(}\PY{l+s+s2}{\PYZdq{}}\PY{l+s+s2}{Svo er f(theta):}\PY{l+s+s2}{\PYZdq{}}\PY{p}{,} \PY{n}{f}\PY{p}{(}\PY{n}{thetas2}\PY{p}{[}\PY{n}{i}\PY{p}{]}\PY{p}{,} \PY{n}{p1}\PY{p}{,} \PY{n}{p2}\PY{p}{,} \PY{n}{p3}\PY{p}{,} 
                                            \PY{n}{L1}\PY{p}{,} \PY{n}{L2}\PY{p}{,} \PY{n}{L3}\PY{p}{,} \PY{n}{gamma}\PY{p}{,} \PY{n}{x1}\PY{p}{,} \PY{n}{x2}\PY{p}{,} \PY{n}{y2}\PY{p}{)}\PY{p}{,} \PY{l+s+s2}{\PYZdq{}}\PY{l+s+se}{\PYZbs{}n}\PY{l+s+s2}{\PYZdq{}}\PY{p}{)}
        
        \PY{c+c1}{\PYZsh{} Geymir x og y gildin fyrir útreiknaðar núllstöðvar f(theta)}
        \PY{n}{xy2} \PY{o}{=} \PY{n}{np}\PY{o}{.}\PY{n}{zeros}\PY{p}{(}\PY{p}{[}\PY{l+m+mi}{6}\PY{p}{,}\PY{l+m+mi}{2}\PY{p}{]}\PY{p}{,} \PY{n}{dtype}\PY{o}{=}\PY{n+nb}{float}\PY{p}{)}
        
        \PY{k}{for} \PY{n}{i} \PY{o+ow}{in} \PY{n+nb}{range}\PY{p}{(}\PY{l+m+mi}{0}\PY{p}{,} \PY{l+m+mi}{6}\PY{p}{)}\PY{p}{:} 
                \PY{n}{xy2}\PY{p}{[}\PY{n}{i}\PY{p}{]} \PY{o}{=} \PY{n}{xyCalc}\PY{p}{(}\PY{n}{thetas2}\PY{p}{[}\PY{n}{i}\PY{p}{]}\PY{p}{,} \PY{n}{p1}\PY{p}{,} \PY{n}{p2}\PY{p}{,} \PY{n}{p3}\PY{p}{,} 
                                \PY{n}{L1}\PY{p}{,} \PY{n}{L2}\PY{p}{,} \PY{n}{L3}\PY{p}{,} \PY{n}{gamma}\PY{p}{,} \PY{n}{x1}\PY{p}{,} \PY{n}{x2}\PY{p}{,} \PY{n}{y2}\PY{p}{)}
        
        \PY{c+c1}{\PYZsh{} Útfrá útreiknuðum theta, x og y gildum getum við síðan reiknað }
        \PY{c+c1}{\PYZsh{} xL2, yL2, xL3, yL3}
        \PY{n}{xy2L23} \PY{o}{=} \PY{n}{np}\PY{o}{.}\PY{n}{zeros}\PY{p}{(}\PY{p}{[}\PY{l+m+mi}{6}\PY{p}{,}\PY{l+m+mi}{4}\PY{p}{]}\PY{p}{,} \PY{n}{dtype}\PY{o}{=}\PY{n+nb}{float}\PY{p}{)}
        
        \PY{k}{for} \PY{n}{i} \PY{o+ow}{in} \PY{n+nb}{range}\PY{p}{(}\PY{l+m+mi}{0}\PY{p}{,} \PY{l+m+mi}{6}\PY{p}{)}\PY{p}{:}
                \PY{n}{xy2L23}\PY{p}{[}\PY{n}{i}\PY{p}{]} \PY{o}{=} \PY{n}{xyLCalc}\PY{p}{(}\PY{n}{thetas2}\PY{p}{[}\PY{n}{i}\PY{p}{]}\PY{p}{,} \PY{n}{xy2}\PY{p}{[}\PY{n}{i}\PY{p}{,}\PY{l+m+mi}{0}\PY{p}{]}\PY{p}{,} \PY{n}{xy2}\PY{p}{[}\PY{n}{i}\PY{p}{,}\PY{l+m+mi}{1}\PY{p}{]}\PY{p}{,} \PY{n}{p1}\PY{p}{,} \PY{n}{p2}\PY{p}{,} \PY{n}{p3}\PY{p}{,} 
                                    \PY{n}{L1}\PY{p}{,} \PY{n}{L2}\PY{p}{,} \PY{n}{L3}\PY{p}{,} \PY{n}{gamma}\PY{p}{,} \PY{n}{x1}\PY{p}{,} \PY{n}{x2}\PY{p}{,} \PY{n}{y2}\PY{p}{)}
        
        \PY{c+c1}{\PYZsh{} Teiknum Stewart platform fyrir þessar sex núllstöðvar f(theta)}
        \PY{c+c1}{\PYZsh{} og könnum í leiðinni hvort lengdirnar á struts séu p1, p2 og p3}
        \PY{k}{for} \PY{n}{i} \PY{o+ow}{in} \PY{n+nb}{range}\PY{p}{(}\PY{l+m+mi}{0}\PY{p}{,} \PY{l+m+mi}{6}\PY{p}{)}\PY{p}{:}
                \PY{n}{p1Calc2} \PY{o}{=} \PY{n}{math}\PY{o}{.}\PY{n}{hypot}\PY{p}{(}\PY{n}{xy2}\PY{p}{[}\PY{n}{i}\PY{p}{,}\PY{l+m+mi}{0}\PY{p}{]} \PY{o}{\PYZhy{}} \PY{l+m+mi}{0}\PY{p}{,} \PY{n}{xy2}\PY{p}{[}\PY{n}{i}\PY{p}{,}\PY{l+m+mi}{1}\PY{p}{]} \PY{o}{\PYZhy{}} \PY{l+m+mi}{0}\PY{p}{)}
                \PY{n}{p2Calc2} \PY{o}{=} \PY{n}{math}\PY{o}{.}\PY{n}{hypot}\PY{p}{(}\PY{n}{xy2L23}\PY{p}{[}\PY{n}{i}\PY{p}{,}\PY{l+m+mi}{2}\PY{p}{]} \PY{o}{\PYZhy{}} \PY{n}{x1}\PY{p}{,} \PY{n}{xy2L23}\PY{p}{[}\PY{n}{i}\PY{p}{,}\PY{l+m+mi}{3}\PY{p}{]} \PY{o}{\PYZhy{}} \PY{l+m+mi}{0}\PY{p}{)}
                \PY{n}{p3Calc2} \PY{o}{=} \PY{n}{math}\PY{o}{.}\PY{n}{hypot}\PY{p}{(}\PY{n}{xy2L23}\PY{p}{[}\PY{n}{i}\PY{p}{,}\PY{l+m+mi}{0}\PY{p}{]} \PY{o}{\PYZhy{}} \PY{n}{x2}\PY{p}{,} \PY{n}{xy2L23}\PY{p}{[}\PY{n}{i}\PY{p}{,}\PY{l+m+mi}{1}\PY{p}{]} \PY{o}{\PYZhy{}} \PY{n}{y2}\PY{p}{)}
                \PY{n+nb}{print}\PY{p}{(}\PY{l+s+s2}{\PYZdq{}}\PY{l+s+s2}{Fyrir núllstöðina}\PY{l+s+s2}{\PYZdq{}}\PY{p}{,} \PY{n}{thetas2}\PY{p}{[}\PY{n}{i}\PY{p}{]}\PY{p}{,} 
                      \PY{l+s+s2}{\PYZdq{}}\PY{l+s+s2}{þá fást eftirfarandi niðurstöður}\PY{l+s+s2}{\PYZdq{}}\PY{p}{)}
                \PY{n+nb}{print}\PY{p}{(}\PY{l+s+s2}{\PYZdq{}}\PY{l+s+s2}{Nú er p1 = 5 en útreiknað gildi er:}\PY{l+s+s2}{\PYZdq{}}\PY{p}{,} \PY{n}{p1Calc2}\PY{p}{)}
                \PY{n+nb}{print}\PY{p}{(}\PY{l+s+s2}{\PYZdq{}}\PY{l+s+s2}{Nú er p2 = 7 en útreiknað gildi er:}\PY{l+s+s2}{\PYZdq{}}\PY{p}{,} \PY{n}{p2Calc2}\PY{p}{)}
                \PY{n+nb}{print}\PY{p}{(}\PY{l+s+s2}{\PYZdq{}}\PY{l+s+s2}{Nú er p3 = 3 en útreiknað gildi er:}\PY{l+s+s2}{\PYZdq{}}\PY{p}{,} \PY{n}{p3Calc2}\PY{p}{)}
                \PY{n+nb}{print}\PY{p}{(}\PY{l+s+s2}{\PYZdq{}}\PY{l+s+se}{\PYZbs{}n}\PY{l+s+s2}{\PYZdq{}}\PY{p}{)}
                \PY{n}{pName} \PY{o}{=} \PY{l+s+s2}{\PYZdq{}}\PY{l+s+s2}{sa5\PYZhy{}}\PY{l+s+s2}{\PYZdq{}}\PY{o}{+}\PY{n+nb}{str}\PY{p}{(}\PY{n}{i}\PY{o}{+}\PY{l+m+mi}{1}\PY{p}{)}
                \PY{n}{plotStewartPlatform}\PY{p}{(}\PY{n}{xy2}\PY{p}{[}\PY{n}{i}\PY{p}{,}\PY{l+m+mi}{0}\PY{p}{]}\PY{p}{,} \PY{n}{xy2}\PY{p}{[}\PY{n}{i}\PY{p}{,}\PY{l+m+mi}{1}\PY{p}{]}\PY{p}{,} \PY{n}{x1}\PY{p}{,} \PY{n}{x2}\PY{p}{,} \PY{n}{y2}\PY{p}{,} \PY{n}{xy2L23}\PY{p}{[}\PY{n}{i}\PY{p}{,} \PY{l+m+mi}{0}\PY{p}{]}\PY{p}{,}
                                    \PY{n}{xy2L23}\PY{p}{[}\PY{n}{i}\PY{p}{,} \PY{l+m+mi}{1}\PY{p}{]}\PY{p}{,} \PY{n}{xy2L23}\PY{p}{[}\PY{n}{i}\PY{p}{,} \PY{l+m+mi}{2}\PY{p}{]}\PY{p}{,} \PY{n}{xy2L23}\PY{p}{[}\PY{n}{i}\PY{p}{,} \PY{l+m+mi}{3}\PY{p}{]}\PY{p}{,} 
                                    \PY{l+s+s2}{\PYZdq{}}\PY{l+s+s2}{Suggested Activity 5}\PY{l+s+s2}{\PYZdq{}}\PY{p}{,} \PY{n}{pName}\PY{p}{)}
\end{Verbatim}


    \begin{center}
    \adjustimage{max size={0.9\linewidth}{0.9\paperheight}}{output_28_0.png}
    \end{center}
    { \hspace*{\fill} \\}
    
    \begin{Verbatim}[commandchars=\\\{\}]
Ef theta er raunveruleg núllstöð þá ætti
f(theta) að vera mjög nálægt núlli 

Nú fæst að theta er: -0.673157486371671
Svo er f(theta): -1.4551915228366852e-11 

Nú fæst að theta er: -0.3547402704156737
Svo er f(theta): 3.637978807091713e-12 

Nú fæst að theta er: 0.03776676057591458
Svo er f(theta): -1.8189894035458565e-12 

Nú fæst að theta er: 0.45887818104896444
Svo er f(theta): -6.923528417246416e-11 

Nú fæst að theta er: 0.9776728950003634
Svo er f(theta): -1.000444171950221e-11 

Nú fæst að theta er: 2.513852799350387
Svo er f(theta): 4.0745362639427185e-10 

Fyrir núllstöðina -0.673157486371671 þá fást eftirfarandi niðurstöður
Nú er p1 = 5 en útreiknað gildi er: 4.999999999999998
Nú er p2 = 7 en útreiknað gildi er: 6.999999999999999
Nú er p3 = 3 en útreiknað gildi er: 2.9999999999999987



    \end{Verbatim}

    \begin{center}
    \adjustimage{max size={0.9\linewidth}{0.9\paperheight}}{output_28_2.png}
    \end{center}
    { \hspace*{\fill} \\}
    
    \begin{Verbatim}[commandchars=\\\{\}]
Fyrir núllstöðina -0.3547402704156737 þá fást eftirfarandi niðurstöður
Nú er p1 = 5 en útreiknað gildi er: 5.0
Nú er p2 = 7 en útreiknað gildi er: 7.0
Nú er p3 = 3 en útreiknað gildi er: 3.0



    \end{Verbatim}

    \begin{center}
    \adjustimage{max size={0.9\linewidth}{0.9\paperheight}}{output_28_4.png}
    \end{center}
    { \hspace*{\fill} \\}
    
    \begin{Verbatim}[commandchars=\\\{\}]
Fyrir núllstöðina 0.03776676057591458 þá fást eftirfarandi niðurstöður
Nú er p1 = 5 en útreiknað gildi er: 4.999999999999998
Nú er p2 = 7 en útreiknað gildi er: 6.999999999999998
Nú er p3 = 3 en útreiknað gildi er: 2.999999999999997



    \end{Verbatim}

    \begin{center}
    \adjustimage{max size={0.9\linewidth}{0.9\paperheight}}{output_28_6.png}
    \end{center}
    { \hspace*{\fill} \\}
    
    \begin{Verbatim}[commandchars=\\\{\}]
Fyrir núllstöðina 0.45887818104896444 þá fást eftirfarandi niðurstöður
Nú er p1 = 5 en útreiknað gildi er: 4.999999999999774
Nú er p2 = 7 en útreiknað gildi er: 6.999999999999839
Nú er p3 = 3 en útreiknað gildi er: 2.9999999999996243



    \end{Verbatim}

    \begin{center}
    \adjustimage{max size={0.9\linewidth}{0.9\paperheight}}{output_28_8.png}
    \end{center}
    { \hspace*{\fill} \\}
    
    \begin{Verbatim}[commandchars=\\\{\}]
Fyrir núllstöðina 0.9776728950003634 þá fást eftirfarandi niðurstöður
Nú er p1 = 5 en útreiknað gildi er: 4.9999999999999964
Nú er p2 = 7 en útreiknað gildi er: 6.999999999999997
Nú er p3 = 3 en útreiknað gildi er: 2.9999999999999947



    \end{Verbatim}

    \begin{center}
    \adjustimage{max size={0.9\linewidth}{0.9\paperheight}}{output_28_10.png}
    \end{center}
    { \hspace*{\fill} \\}
    
    \begin{Verbatim}[commandchars=\\\{\}]
Fyrir núllstöðina 2.513852799350387 þá fást eftirfarandi niðurstöður
Nú er p1 = 5 en útreiknað gildi er: 5.000000000000003
Nú er p2 = 7 en útreiknað gildi er: 7.000000000000002
Nú er p3 = 3 en útreiknað gildi er: 3.0000000000000053



    \end{Verbatim}

    \begin{center}
    \adjustimage{max size={0.9\linewidth}{0.9\paperheight}}{output_28_12.png}
    \end{center}
    { \hspace*{\fill} \\}
    
    \subsection{Suggested Activity 6}\label{suggested-activity-6}

Find a strut length \(p_2\), with the rest of the parameters as in Step
4, for which there are only two poses.

    \begin{Verbatim}[commandchars=\\\{\}]
{\color{incolor}In [{\color{incolor} }]:} \PY{c+c1}{\PYZsh{} Stikar úr Suggested Activity 4}
        \PY{n}{x1} \PY{o}{=} \PY{l+m+mi}{5} 
        \PY{n}{x2}\PY{p}{,} \PY{n}{y2} \PY{o}{=} \PY{l+m+mi}{0}\PY{p}{,} \PY{l+m+mi}{6}
        \PY{n}{L1} \PY{o}{=} \PY{n}{L3} \PY{o}{=} \PY{l+m+mi}{3} 
        \PY{n}{L2} \PY{o}{=} \PY{l+m+mi}{3}\PY{o}{*}\PY{n}{np}\PY{o}{.}\PY{n}{sqrt}\PY{p}{(}\PY{l+m+mi}{2}\PY{p}{)}
        \PY{n}{gamma} \PY{o}{=} \PY{n}{np}\PY{o}{.}\PY{n}{pi}\PY{o}{/}\PY{l+m+mi}{4}
        \PY{n}{p1} \PY{o}{=} \PY{l+m+mi}{5}
        \PY{n}{p3} \PY{o}{=} \PY{l+m+mi}{3}
        
        \PY{c+c1}{\PYZsh{} Fall sem telur fjölda róta fallsins f(theta)}
        \PY{c+c1}{\PYZsh{} fyrir gefna stika}
        \PY{k}{def} \PY{n+nf}{countRoots}\PY{p}{(}\PY{n}{f}\PY{p}{,} \PY{n}{p2}\PY{p}{)}\PY{p}{:} 
                \PY{n}{deltaTheta} \PY{o}{=} \PY{l+m+mf}{0.001}
                \PY{n}{thetaOld} \PY{o}{=} \PY{o}{\PYZhy{}}\PY{n}{np}\PY{o}{.}\PY{n}{pi}
                \PY{n}{oldSign} \PY{o}{=} \PY{n}{f}\PY{p}{(}\PY{n}{thetaOld}\PY{p}{,} \PY{n}{p1}\PY{p}{,} \PY{n}{p2}\PY{p}{,} \PY{n}{p3}\PY{p}{,} 
                            \PY{n}{L1}\PY{p}{,} \PY{n}{L2}\PY{p}{,} \PY{n}{L3}\PY{p}{,} \PY{n}{gamma}\PY{p}{,} \PY{n}{x1}\PY{p}{,} \PY{n}{x2}\PY{p}{,} \PY{n}{y2}\PY{p}{)} \PY{o}{\PYZgt{}} \PY{l+m+mi}{0}
                \PY{n}{count} \PY{o}{=} \PY{l+m+mi}{0}
                \PY{k}{for} \PY{n}{theta} \PY{o+ow}{in} \PY{n}{np}\PY{o}{.}\PY{n}{arange}\PY{p}{(}\PY{o}{\PYZhy{}}\PY{n}{np}\PY{o}{.}\PY{n}{pi}\PY{p}{,} \PY{n}{np}\PY{o}{.}\PY{n}{pi}\PY{o}{+}\PY{n}{deltaTheta}\PY{p}{,} \PY{n}{deltaTheta}\PY{p}{)}\PY{p}{:}
                        \PY{n}{thetaNew} \PY{o}{=} \PY{n}{theta}
                        \PY{n}{newSign} \PY{o}{=} \PY{n}{f}\PY{p}{(}\PY{n}{thetaNew}\PY{p}{,} \PY{n}{p1}\PY{p}{,} \PY{n}{p2}\PY{p}{,} \PY{n}{p3}\PY{p}{,} 
                                    \PY{n}{L1}\PY{p}{,} \PY{n}{L2}\PY{p}{,} \PY{n}{L3}\PY{p}{,} \PY{n}{gamma}\PY{p}{,} \PY{n}{x1}\PY{p}{,} \PY{n}{x2}\PY{p}{,} \PY{n}{y2}\PY{p}{)} \PY{o}{\PYZgt{}} \PY{l+m+mi}{0}
                        \PY{k}{if}\PY{p}{(}\PY{n}{newSign} \PY{o}{!=} \PY{n}{oldSign}\PY{p}{)}\PY{p}{:}
                                \PY{n}{count} \PY{o}{+}\PY{o}{=} \PY{l+m+mi}{1}
                        \PY{n}{oldSign} \PY{o}{=} \PY{n}{newSign}
                \PY{k}{return} \PY{n}{count}
\end{Verbatim}


    Höfum séð að það eru 4 rætur og þar með 4 stöður þegar \(p_2 = 5\) Höfum
séð að það eru 6 rætur og þar með 6 stöður þegar \(p_2 = 7\) Skoðum
gildi þar í kring. Vitum að það eru (líklegast) formerkjaskipti í rótum.
Einnig þar sem \(p_2\) er mælikvarði á lengd vitum við að \(p>0\). Því
leitum við að formerkjaskiptum á bilinu {[}\$0,12{]}\$

    \begin{Verbatim}[commandchars=\\\{\}]
{\color{incolor}In [{\color{incolor} }]:} \PY{c+c1}{\PYZsh{} Prófum mismunandi gildi á p2}
        \PY{k}{def} \PY{n+nf}{p2Roots}\PY{p}{(}\PY{n}{p2Min}\PY{p}{,} \PY{n}{p2Max}\PY{p}{,} \PY{n}{interval}\PY{p}{,} \PY{n}{noRoots}\PY{p}{)}\PY{p}{:} 
          \PY{n}{p2Array} \PY{o}{=} \PY{n}{np}\PY{o}{.}\PY{n}{arange}\PY{p}{(}\PY{n}{p2Min}\PY{p}{,} \PY{n}{p2Max}\PY{o}{+}\PY{n}{interval}\PY{p}{,} \PY{n}{interval}\PY{p}{)}
          \PY{n}{p2Roots} \PY{o}{=} \PY{n+nb}{list}\PY{p}{(}\PY{p}{)}
          \PY{k}{for} \PY{n}{i} \PY{o+ow}{in} \PY{n}{p2Array}\PY{p}{:}
                  \PY{k}{if}\PY{p}{(}\PY{n}{countRoots}\PY{p}{(}\PY{n}{f}\PY{p}{,} \PY{n}{i}\PY{p}{)} \PY{o}{==} \PY{n}{noRoots}\PY{p}{)}\PY{p}{:}
                          \PY{n}{p2Roots}\PY{o}{.}\PY{n}{append}\PY{p}{(}\PY{n}{i}\PY{p}{)}
        
          \PY{k}{for} \PY{n}{i} \PY{o+ow}{in} \PY{n+nb}{range}\PY{p}{(}\PY{n+nb}{len}\PY{p}{(}\PY{n}{p2Roots}\PY{p}{)}\PY{p}{)}\PY{p}{:}
            \PY{n+nb}{print}\PY{p}{(}\PY{n}{p2Roots}\PY{p}{[}\PY{n}{i}\PY{p}{]}\PY{p}{)}
        
        \PY{n}{p2Roots}\PY{p}{(}\PY{l+m+mi}{0}\PY{p}{,} \PY{l+m+mi}{12}\PY{p}{,} \PY{l+m+mf}{0.1}\PY{p}{,} \PY{l+m+mi}{2}\PY{p}{)}
\end{Verbatim}


    \begin{Verbatim}[commandchars=\\\{\}]
3.8000000000000003
3.9000000000000004
4.0
4.1000000000000005
4.2
4.3
4.4
4.5
4.6000000000000005
4.7
4.800000000000001
7.9
8.0
8.1
8.200000000000001
8.3
8.4
8.5
8.6
8.700000000000001
8.8
8.9
9.0
9.1
9.200000000000001

    \end{Verbatim}

    Við sjáum að fyrir \(p_2 \in [3.8, 4.8] \text{ } \cup [7.9, 9.2]\) þá
fást 2 stöður.

    \subsection{Suggested Activity 7}\label{suggested-activity-7}

Calculate the intervals in \(p_2\), with the rest of the parameters as
in Step 4, for which there are 0, 2, 4 and 6 poses, respectively

    Byrjum á að plotta graf með gildinu á \(p_2\) á x - ás og fjölda staða á
y - ás. Liðir Stewart platform virki sem skyldi þá er eðlilegt að gera
ráð fyrir því að einum af stoðunum (strut) séu margfalt lengri en hinir.
Því látum við hámarksgildi \(p_2\) vera 100 eða tuttuguföld stærð
næstlengstu stoðarinnar, \(p_1\).

    \begin{Verbatim}[commandchars=\\\{\}]
{\color{incolor}In [{\color{incolor} }]:} \PY{c+c1}{\PYZsh{} Stikar úr Suggested Activity 4}
        \PY{n}{x1} \PY{o}{=} \PY{l+m+mi}{5} 
        \PY{n}{x2}\PY{p}{,} \PY{n}{y2} \PY{o}{=} \PY{l+m+mi}{0}\PY{p}{,} \PY{l+m+mi}{6}
        \PY{n}{L1} \PY{o}{=} \PY{n}{L3} \PY{o}{=} \PY{l+m+mi}{3} 
        \PY{n}{L2} \PY{o}{=} \PY{l+m+mi}{3}\PY{o}{*}\PY{n}{np}\PY{o}{.}\PY{n}{sqrt}\PY{p}{(}\PY{l+m+mi}{2}\PY{p}{)}
        \PY{n}{gamma} \PY{o}{=} \PY{n}{np}\PY{o}{.}\PY{n}{pi}\PY{o}{/}\PY{l+m+mi}{4}
        \PY{n}{p1} \PY{o}{=} \PY{l+m+mi}{5}
        \PY{n}{p3} \PY{o}{=} \PY{l+m+mi}{3}
        
        \PY{n}{maxVal} \PY{o}{=} \PY{l+m+mi}{100}
        \PY{n}{interval} \PY{o}{=} \PY{l+m+mf}{0.1}
        \PY{n}{no} \PY{o}{=} \PY{n+nb}{int}\PY{p}{(}\PY{n}{maxVal}\PY{o}{/}\PY{n}{interval}\PY{p}{)}
        \PY{n}{noSolutions} \PY{o}{=} \PY{n}{np}\PY{o}{.}\PY{n}{zeros}\PY{p}{(}\PY{n}{no}\PY{p}{)}
        \PY{n}{p2} \PY{o}{=} \PY{l+m+mi}{0}
        \PY{k}{for} \PY{n}{i} \PY{o+ow}{in} \PY{n+nb}{range}\PY{p}{(}\PY{n+nb}{len}\PY{p}{(}\PY{n}{noSolutions}\PY{p}{)}\PY{p}{)}\PY{p}{:}
          \PY{n}{noSolutions}\PY{p}{[}\PY{n}{i}\PY{p}{]} \PY{o}{=} \PY{n}{countRoots}\PY{p}{(}\PY{n}{f}\PY{p}{,} \PY{n}{p2}\PY{p}{)}
          \PY{n}{p2} \PY{o}{+}\PY{o}{=} \PY{n}{interval}
        
        \PY{n}{theRange} \PY{o}{=} \PY{n}{np}\PY{o}{.}\PY{n}{arange}\PY{p}{(}\PY{l+m+mi}{0}\PY{p}{,} \PY{n}{maxVal}\PY{p}{,} \PY{n}{interval}\PY{p}{)}
        \PY{n}{matrixRange} \PY{o}{=} \PY{n}{np}\PY{o}{.}\PY{n}{arange}\PY{p}{(}\PY{l+m+mi}{0}\PY{p}{,} \PY{n+nb}{len}\PY{p}{(}\PY{n}{noSolutions}\PY{p}{)}\PY{p}{,} \PY{l+m+mi}{1}\PY{p}{)}
        
        \PY{n}{figure}\PY{p}{,} \PY{n}{axis} \PY{o}{=} \PY{n}{plt}\PY{o}{.}\PY{n}{subplots}\PY{p}{(}\PY{n}{figsize}\PY{o}{=}\PY{p}{(}\PY{l+m+mi}{15}\PY{p}{,} \PY{l+m+mi}{6}\PY{p}{)}\PY{p}{)}
        
        \PY{c+c1}{\PYZsh{} Teiknum grafið}
        \PY{n}{axis}\PY{o}{.}\PY{n}{scatter}\PY{p}{(}\PY{n}{theRange}\PY{p}{,} \PY{n}{noSolutions}\PY{p}{[}\PY{n}{matrixRange}\PY{p}{]}\PY{p}{)}
        
        \PY{c+c1}{\PYZsh{} Setjum titil á grafið og ásana}
        \PY{n}{axis}\PY{o}{.}\PY{n}{set}\PY{p}{(}\PY{n}{xlabel}\PY{o}{=}\PY{l+s+sa}{r}\PY{l+s+s2}{\PYZdq{}}\PY{l+s+s2}{\PYZdl{}}\PY{l+s+si}{\PYZob{}p\PYZus{}2\PYZcb{}}\PY{l+s+s2}{\PYZdl{}}\PY{l+s+s2}{\PYZdq{}}\PY{p}{,} 
                        \PY{n}{ylabel}\PY{o}{=}\PY{l+s+s2}{\PYZdq{}}\PY{l+s+s2}{Fjöldi staða (núllstöðva)}\PY{l+s+s2}{\PYZdq{}}\PY{p}{,}
                        \PY{n}{title} \PY{o}{=} \PY{l+s+s2}{\PYZdq{}}\PY{l+s+s2}{\PYZdq{}}\PY{p}{)}
        \PY{n}{axis}\PY{o}{.}\PY{n}{set\PYZus{}ylim}\PY{p}{(}\PY{p}{[}\PY{l+m+mi}{0}\PY{p}{,}\PY{l+m+mi}{6}\PY{p}{]}\PY{p}{)}
        \PY{n}{axis}\PY{o}{.}\PY{n}{grid}\PY{p}{(}\PY{p}{)}
        
        \PY{c+c1}{\PYZsh{} Setjum minnsta sýnilega bilið á x \PYZhy{} ásnum}
        \PY{n}{plt}\PY{o}{.}\PY{n}{xticks}\PY{p}{(}\PY{n}{np}\PY{o}{.}\PY{n}{arange}\PY{p}{(}\PY{l+m+mi}{0}\PY{p}{,} \PY{n}{maxVal}\PY{o}{+}\PY{l+m+mi}{1}\PY{p}{,} \PY{l+m+mi}{10}\PY{p}{)}\PY{p}{)}
        \PY{n}{plt}\PY{o}{.}\PY{n}{yticks}\PY{p}{(}\PY{n}{np}\PY{o}{.}\PY{n}{arange}\PY{p}{(}\PY{l+m+mi}{0}\PY{p}{,} \PY{l+m+mi}{8}\PY{p}{,} \PY{l+m+mi}{1}\PY{p}{)}\PY{p}{)}
        \PY{c+c1}{\PYZsh{} Vistum myndina}
        \PY{n}{fileString} \PY{o}{=} \PY{l+s+s2}{\PYZdq{}}\PY{l+s+s2}{sa7}\PY{l+s+s2}{\PYZdq{}} \PY{o}{+} \PY{l+s+s2}{\PYZdq{}}\PY{l+s+s2}{.png}\PY{l+s+s2}{\PYZdq{}}
        \PY{n}{figure}\PY{o}{.}\PY{n}{savefig}\PY{p}{(}\PY{n}{fileString}\PY{p}{)}
        \PY{n}{plt}\PY{o}{.}\PY{n}{grid}\PY{p}{(}\PY{k+kc}{True}\PY{p}{)}
        \PY{n}{plt}\PY{o}{.}\PY{n}{show}\PY{p}{(}\PY{p}{)}
\end{Verbatim}


    \begin{center}
    \adjustimage{max size={0.9\linewidth}{0.9\paperheight}}{output_36_0.png}
    \end{center}
    { \hspace*{\fill} \\}
    
    Við sjáum að það fást 0 stöður þegar \(10 <p_2 \leq 100\). Til að finna
hvar fjöldi staða er ekki núll, takmörkum okkur þá við bilið
\(0 < p_2 \leq 10\).

    Úr Suggested Activity 6 höfm við að það fást 2 stöður þegar
\(p_2 \in [3.8, 4.8] \text{ } \cup [7.9, 9.2]\)

    \begin{Verbatim}[commandchars=\\\{\}]
{\color{incolor}In [{\color{incolor} }]:} \PY{n}{p2Roots}\PY{p}{(}\PY{l+m+mi}{0}\PY{p}{,} \PY{l+m+mi}{10}\PY{p}{,} \PY{l+m+mf}{0.1}\PY{p}{,} \PY{l+m+mi}{4}\PY{p}{)}
\end{Verbatim}


    \begin{Verbatim}[commandchars=\\\{\}]
4.9
5.0
5.1000000000000005
5.2
5.300000000000001
5.4
5.5
5.6000000000000005
5.7
5.800000000000001
5.9
6.0
6.1000000000000005
6.2
6.300000000000001
6.4
6.5
6.6000000000000005
6.7
6.800000000000001
6.9
7.1000000000000005
7.2
7.300000000000001
7.4
7.5
7.6000000000000005
7.7
7.800000000000001

    \end{Verbatim}

    Við sjáum að það fást 4 stöður þegar
\(p_2 \in [4.9, 6.9] \text{ } \cup [7.1, 7.8]\)

    \begin{Verbatim}[commandchars=\\\{\}]
{\color{incolor}In [{\color{incolor} }]:} \PY{n}{p2Roots}\PY{p}{(}\PY{l+m+mi}{0}\PY{p}{,} \PY{l+m+mi}{10}\PY{p}{,} \PY{l+m+mf}{0.1}\PY{p}{,} \PY{l+m+mi}{6}\PY{p}{)}
\end{Verbatim}


    \begin{Verbatim}[commandchars=\\\{\}]
7.0

    \end{Verbatim}

    Við sjáum að það fást 6 stöður þegar \(p_2 = 7\)

    \begin{Verbatim}[commandchars=\\\{\}]
{\color{incolor}In [{\color{incolor} }]:} \PY{n}{p2Roots}\PY{p}{(}\PY{l+m+mi}{0}\PY{p}{,} \PY{l+m+mi}{10}\PY{p}{,} \PY{l+m+mf}{0.1}\PY{p}{,} \PY{l+m+mi}{0}\PY{p}{)}
\end{Verbatim}


    \begin{Verbatim}[commandchars=\\\{\}]
0.0
0.1
0.2
0.30000000000000004
0.4
0.5
0.6000000000000001
0.7000000000000001
0.8
0.9
1.0
1.1
1.2000000000000002
1.3
1.4000000000000001
1.5
1.6
1.7000000000000002
1.8
1.9000000000000001
2.0
2.1
2.2
2.3000000000000003
2.4000000000000004
2.5
2.6
2.7
2.8000000000000003
2.9000000000000004
3.0
3.1
3.2
3.3000000000000003
3.4000000000000004
3.5
3.6
3.7
9.3
9.4
9.5
9.600000000000001
9.700000000000001
9.8
9.9
10.0

    \end{Verbatim}

    Við sjáum að það fást 0 stöður þegar
\(p_2 \in [0, 3.7] \text{ } \cup [9.3, 100]\)

    \subsection{Varðandi leit að núllstöðvum í Suggested Activity 2, 4 og
5}\label{varuxf0andi-leit-auxf0-nuxfallstuxf6uxf0vum-uxed-suggested-activity-2-4-og-5}

Í þessum þremur æfingum átti að finna núllstöðvar \(f(\theta)\) á bili
\$ \theta \in[-\pi,\pi]\$. Fyrsta skrefið við að finna núllstöðvarnar
var að plotta ferilinn fyrir öll \(\theta \in [-\pi, \pi]\). Í öllum
æfingunum var ferilinn \(f(\theta)\) er mjög flatur í kringum
\(f(\theta) = 0\), sem bendir til þess að villumögnunin (e. error
magnification factor) sé mikil á þessu svæði. Þannig þó \(|f(\theta)|\)
sé mjög lág tala fyrir eitthvað \(\theta\) þá getum við samt verið
hlutfallslega mjög langt frá hinni réttu rót \(f(\theta)\). Því þarf
sérstaklega að gæta þess að stop skilyrðið sé ekki of slakt við leit
rótarinnar. Fallið fsolve

    \subsection{Suggested Activity 8}\label{suggested-activity-8}

\emph{Derive or look up the equations representing the forward
kinematics of the three-dimensional, six-degrees-of-freedom Stewart
platform. Write a Matlab program and demonstrate its use to solve the
forward kinematics.}

Verkefnið snýst um að hanna sex-leggja sviðspall. Til eru mismunandi
lausnir á verkefninu.
\href{https://www.youtube.com/watch?v=PIl8Epy9xIw\&t=34s}{Ein lausn} er
þannig að hver leggur er í tveimur pörtum, báðir af fastri lengd, annar
í snertingu við sviðspallinn, og hinn hluti hvers leggs snýst um sinn
eigin snúningsöxul.
\href{https://www.youtube.com/watch?v=PIl8Epy9xIw\&t=34s}{Önnur lausn}
er þannig að leggirnir geta lengst, en upphafshæð þeirra er föst.

Skoðum aðeins
\href{https://www.researchgate.net/publication/2564027_Forward_Kinematics_of_a_Stewart_Platform_Mechanism}{verkefni}
sem tekst á við seinni útgáfuna. Sá hluti verkefnisins sem snýr að
framvirkri hreyfifræði (e. forward kinematics) snýst um að finna
staðsetningu og halla sviðspallsins þegar lengdir leggjanna eru þekktar,
svipað því sem er var gert í tvívíða þriggja-leggja verkefninu að ofan.
Það er engin góð þekkt jafna til að leysa það verkefni, nema bara að
finna núllstöðvar á jöfnu svipaðri og var gert að ofan, þ.e.a.s. fundin
er bestunarlausn með reikniritum. Látum vigurinn
\[\bar{t} = [t_x, t_y, t_z]\] tákna staðsetningu miðju sviðspallsins í
þrívídd, og þá eru skilgreind þrjú snúningshorn pallsins
\(\alpha, \beta, \gamma\) um \(x, y, z\) ásana. Svo eru leggirnir
skilgreindir með sex vigrum af fastri lengd, og sex vigrar af
breytilegri lengd (því leggirnir eru lengjanlegir) og staðsetningarhnit
þeirra eru \[\bar{b_i} = [b_ix, b_iy, 0]\] og
\[\bar{p_i} = [p_ix, p_iy, 0]\] fyrir \[i = 1,...,6\] \(z\)-hnitin eru
öll \(0\) því reikningar miða við að grunnhnit leggjanna séu á jörðinni
eða upphafspunkti z-áss. Nú er hægt að tákna lengdir leggjanna
\(\bar{l_i} = -\bar{b_i} + \bar{t} + \underline{R}\cdot\bar{p_i}\) þar
sem \$ \underline{R}\$ er snúningsfylki, reiknað út frá snúningshornunum
þremur. Farið er nánar út í jöfnurnar á blaðsíðum 4,5 og 6 í verkefninu
í hlekknum að ofan.

Svo er staðsetning og halli sviðspallsins fyrir allar mismunandi lengdir
leggjanna reiknaður og þar með hægt að kortleggja hreyfingu hans í
þremur víddum.


    % Add a bibliography block to the postdoc
    
    
    
    \end{document}
